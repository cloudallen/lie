\section{Representations}
Letting $\g$ be a semisimple Lie Algebra with cartan subalgebra $\h$ and $V$ any finite-dimensional $\g$-module, we've seen that $\h$ acts semisimply on $V$, this allows us to decompose $V$ into eigenspaces with respect to $\h$.
$$ V = \bigoplus_{\lambda \in \h^*} V_\lambda$$
Where $V_\lambda=\{v \in V| H\cdot v=\lambda(H)v \text{ for all } H \in h\}$, we'll call it a weight space whenever $V_\lambda \not= 0$	and we'll call $\lambda$ a weight of $V$.\\
This relates very deeply to the previous discussion on weights on root systems, as we can examplify in the following:
\begin{ex}
Let $V=\g$ with the adjoint action, then the weights $\lambda$ are the roots $\alpha$, along with $0$, where $\h$ is the $0$-weight space.\\
Letting $\g=\gsl(2,\F)$, then $\h^*$ is one-dimensional, and the weights of any particular representation were subsets of the lattice of integers.\\
\end{ex}
If $\dim V = \infty$, there is no assurance that such decomposition holds, nevertheless as weights spaces are, in some sense, eigenspaces of eigenspaces the sum is always direct.\\
To formalize the concept of weights on arbitrary Lie Algebras and representations, we'll generalize the results of additivity under the $\g$-action of weights already seen in the case of $\gsl_2$ and the adjoint action.\\
\begin{lema}
If $V$ is an arbitrary $\g$-module. Then:
\begin{enumerate}[label=(\alph*)]
\item $\g_\alpha$ maps $V_\lambda$ to $V_{\lambda+\alpha}$($\alpha \in \Phi$ and $\lambda \in \h^*$)
\item The sum $V' = \sum_{\lambda \in \h^*} V_{\lambda}$ is direct, and $V'$ is a submodule of $V$
\item If $V$ is finite-dimensional, then $V=V'$
\end{enumerate}
\end{lema}
\begin{proof}
Let $X \in \g_{\alpha}$, then $H(Xv) = X(Hv) + [H,X]v = \lambda(H)Xv + \alpha(H)Xv = (\lambda+\alpha)(H)v$.\\
Since $\g$ is the sum of root-spaces, then $V'$ is closed under the action of $\g$ by the previous note, as it is also a sum of vector spaces it is a vector space.\\
Finally, if $V$ is finite-dimensional, since we are taking an algebraically closed field then it decomposes as sum of eigenspaces.\\
\end{proof}\\
As we did with $\gsl_2$, we shall define a maximal vector, let $\Delta \subset \Phi$ be a basis of the root system, and $\Phi^+$, $\Phi^-$ be the positive and negative roots.\\
Given $V$ a $\g$-module, we call $v^+$ a maximal vector of weight $\lambda$ if $\g_\alpha v = 0$ for any $\alpha \in \Phi^+$ and $v = V_\lambda$. If $\dim V = \infty$, there is no reason for a maximal vector to exist, but on every finite-dimensional representation, they have to exist because the algebra $\h \oplus n_+$ is solvable , so the existence of a common eigenvector is necessary by Lie's Theorem, and the eigenvalues of $n_+$ as nilpotent elements has to be $0$.\\
In order to study finite dimensional representations, it is useful to think about a large class of representations given by standard cyclic of a given weight, where $\lambda$ is called the highest weight, we call $V$ standard cyclic if $V=\g\cdot v^+$ with a maximal vector $v^+$ of weight $\lambda$.To justify this terminology, let's look at the following results:
\begin{teo}
Let $V$ be a standard cyclic $\g$-module with maximal vector $v^+ \in V_\lambda$. Also let $\Phi^+ = \{\beta_1,\cdots,\beta_m\}$, $\Delta = \{\alpha_1,\cdots,\alpha_\ell\}$ ,$X_k$ the vector that spans $g_{\beta_k}$ and $Y_k$ spans $g_{-\beta_k}$. Then:
\begin{enumerate}[label=(\alph*)]
\label{42standardcyclic}
\item $V$ is spanned by the vectors $Y_1^{i_1}\cdots Y_m^{i_m}v^+\ \ (i_k \in \mathbb{Z}_{\ge 0})$, in particular $V$ is a direct sum of weight spaces.
\item The weights of $V$ are of the form $\mu = \lambda - \sum_{i=1}^\ell k_i\alpha_i$, with $k_i \in \mathbb{Z}_{\ge 0}$ 
\item For each $\mu \in \h^*$ we have $\dim V_\mu < \infty$ and $\dim V_\lambda = 1$
\item Each submodule of $V$ is a direct sum of its weight spaces.
\item V is an indecomposable $\g$-module, with a unique maximal(proper) submodule and a corresponding unique irreducible quotient.
\item Every non-zero homomorphic image of $V$ is also a standard cyclic module of weight $\lambda$.
\end{enumerate}
\end{teo}
\begin{proof}
\begin{enumerate}[label=(\alph*)]
	\item Since $\g=\n_{-} \oplus \h \oplus \n_{+}$ then $U(\g) = U(\n_-)U(\h)U(\n_{+})$ since $v^+$ is a maximal vector then $U(\n_{+})v^+=0$ for any non-unital element in $U(\n_{+})$, and since it's a weight vector $U(\h)v = \lambda(H)v$. Therefore $V=U(\g)v^+ = U(\n_-)v^+$ and $U(\n_-)$ is generated by the ordered monomials by the PBW-theorem applied to the algebra $\n_-$ 
	\item Since every weight $\beta_i$ is positive then they are written as a sum of simple roots, and by $(a)$ every element of $V$ is the image of the action of negative root on the vector $v^+$.
	\item Only a finite number of combinations of the roots $-\beta_m$ centered at $\lambda$ can give rise to a particular weight $\mu = \lambda - \sum_{i=1}^\ell k_i \alpha_l$, in view of $(a)$ they span this weight space, and therefore $\dim V_\mu < \infty$, also in view of $(a)$ there is only one way get the weight $\mu$ with those combinations, which is through $i_k=0$ for all $k$, therefore $\dim V_\lambda = 1$
	\item Since $V$ is a direct sum of weight spaces by $(a)$, then it rests to prove that if a submodule contains a sum then they must contain the weight vectors.\\
	Let $W \subset V$ be a submodule and $w = \sum_{i=1}^r v_i$ for $v_i$ weight vectors of weight $\mu_i$. If some $v_i$ lie in $W$ then the element removing those also lie in $W$ and therefore we may assume $w = v_1 + \cdots + v_k$ with none on $W$ and minimal on $k$, choose $H \in \h$ in a way that $\mu_1(H) \not= \mu_2(H)$, then $Hw \in W$ and so is $\mu_1(H)w$. Therefore:
	$$(H - \mu_1(H)I)w = (\mu_2(H)-\mu_1(H))v_2 + \cdots + (\mu_k(H)-\mu_1(H))v_k \not=0 $$
	Which means one of those elements lie in $W$ by minimality, absurd.
	\item  Since $v^+$ generates $V$ then every proper submodule can't contain it, therefore the sum of all proper submodules is still a proper submodule and trivially maximal, let this be $W$.Since every proper submodule is contained in $W$, proving $V$ can't be decomposed.\\
	Moreover, since $W$ is unique and maximal, there is a unique irreducible quotient $V/W$
	\item Let $\phi(V)$ be such non-zero homomorphic image, then $\phi(v^+)$ generates the image, and by preservation it is also maximal of weight $\lambda$.
\end{enumerate}
\end{proof}
\begin{corol}
	Let $V$ be an irreducible standard cyclic $\g$-module with maximal vector $v^+$ of weight $\lambda$, then there is no other maximal vector(up to scalar multiples) in $V$.
\end{corol}
\begin{proof}
	If $w^+$ is another maximal vector, then $U(\g)w^+ = V$ since $V$ is irreducible. If $\mu$ is the weight of $w^+$ then part $(b)$ applies to both $\mu$ and $\lambda$, and therefore $\mu=\lambda$ and by part $(c)$ they must be proportional.
\end{proof}\\
Based on the previous corollary, one might naturally think about how to describe all the irreducible modules of that form by their highest weight, in fact given a $\lambda \in \h^*$ there always exist an irreducible module with $\lambda$ as the highest weight $\cite{humphreys1}$, furthermore, they are always isomorphic.
\begin{teo}
	If $V,W$ are irreducible standard cyclic of highest weight $\lambda$, then they are isomorphic.
\end{teo}
\begin{proof}
	Let $M = V \oplus W$ and $v^+$, $w^+$ be their respective maximal vectors, then $m^+ = (v^+,w^+)$ is also a maximal vector by how we extend the $\g$-action on $M$.\\
	Let $N = U(\g)m^+$, it is then a standard cyclic submodule of $M$, then the projections from $N$ to the initial modules form irreducible quotients of $N$ and are therefore isomorphic by Theorem \ref{42standardcyclic}$(e)$
\end{proof}
