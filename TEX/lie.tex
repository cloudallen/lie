\documentclass[svgnames,12pt,oneside, openright,a4paper]{scrbook}
\usepackage{tikz,blindtext} %%%% Usado para o titulo  de capitulos
\usepackage{palatino}
\usepackage{chngpage,calc}
\usepackage{hyperref}
\usepackage{kpfonts}
\usepackage{enumitem}
\usepackage{conf}
\usepackage{tikz-cd}
\usepackage[utf8]{inputenc}
\usepackage[english]{babel} %%% Define o Idioma
\usepackage{setspace}
\onehalfspacing
\newenvironment{palavraschaves}{\vspace{1cm}\textbf{Palavras Chaves: }}{}
\newenvironment{keywords}{\vspace{1cm}\textbf{Keywords: }}{}
%%%%%%%%%%%%%%%%%%%%%%%%%%%%Margens
\addtolength{\parindent}{-0.2cm}
\setlength{\textheight}{23cm}
\setlength{\textwidth}{15.5cm}
%%%%%%%%%%%%%%%%% Define o Titulo, o autor e o orientador%%%%%
\newcommand{\titulo}{An Introduction to BGG Category $\bggO$}
\newcommand{\orientador}{Dr. Luis Enrique Ramirez}
\newcommand{\autor}{David José de Melo Júnior\xspace}
\newcommand{\data}{July 2020}
\newcommand{\gl}{\mathfrak{gl}}
\newcommand{\g}{\mathfrak{g}}
\newcommand{\h}{\mathfrak{h}}
\newcommand{\tr}{\text{tr}}
\newcommand{\gsl}{\mathfrak{sl}}
\newcommand{\ad}{\text{ad}}
\newcommand{\A}{\mathcal{A}}
\newcommand{\Aut}{\text{Aut }}
\newcommand{\B}{\mathcal{B}}
\newcommand{\U}{\mathcal{U}}
\newcommand{\z}{\mathfrak{z}}	
\newcommand{\ideal}{\trianglelefteq}
\newcommand{\T}{\mathcal{T}}
\newcommand{\Sim}{\mathcal{S}}
\newcommand{\Tr}{\text{Tr}}
\newcommand{\n}{\mathfrak{n}}
\newcommand{\F}{\mathbb{F}}
\newcommand{\Rad}{\text{Rad }}
\newcommand{\bggO}{\mathcal{O}}
\newcommand{\W}{\mathcal{W}}
\newcommand{\cartan}[2]{\langle {#1},{#2}^\lor\rangle}
\usepackage{amssymb,amsmath,theorem}

%%%%%%Teoremas e outros Ambientes
%% TEOREMAS E OUTROS AMBIENTES

%%%%% TEOREMAS
\newtheorem{teo}{Theorem}[section]
\newtheorem{corol}[teo]{Corolary}
\newtheorem{ddefi}[teo]{Definition}
\newenvironment{definicao} % quadrado 
  {\begin{shaded} \begin{ddefi}\rm}{\end{ddefi} \end{shaded}}
\newenvironment{defi} % sem quadrado s
  {\begin{ddefi}}{\end{ddefi}}
\newtheorem{prop}[teo]{Proposition}
\newtheorem{ex}[teo]{Example}
\newtheorem{remark}[teo]{Remark}
\newtheorem{propri}[teo]{Propriedade}
\newtheorem{lema}[teo]{Lemma}
\newtheorem{obss}[teo]{Observação}
\newtheorem{obss1}[teo]{Observações}
\newenvironment{obs}
  {\begin{obss}\rm}{\end{obss}}
  \newenvironment{observacoes}
  {\begin{obss1}\rm}{\end{obss1}}
\newtheorem{conj}[teo]{Conjectura}
%\newcounter{exemplocount}[chapter]
\newenvironment{exx}
  {\vspace{0.5cm} \refstepcounter{teo} \noindent  \textbf{Exemplo \arabic{chapter}.\arabic{teo}}}{\vspace{0.2cm} }

\theorembodyfont{\rm}



% Demonstra��o (original hspace is 133mm)
\newenvironment{proof}[1][Proof]{\noindent\textbf{#1:} }{\hfill $\Box$}









\begin{document}
%%%Insere o Titulo
\input{titulo}
%%% Insere o Sum�rio
 \tableofcontents
%%%%%%%%%%%%%%%%%
\chapter*{Introduction}
This project serves as an introduction to all the main basic results present in Lie Algebras with an emphasis on representation theory as a gateway to further development of the field, was initially started for the scientific initiation and masters project (PICME) whose purpose is to direct people to mathematics as an introduction to research and logical reasoning.
\chapter{Lie Algebras: Basics}
\section{Lie Algebras as Exponentials of Matrix Lie Groups}
Lie Theory is a huge field in modern mathematical research with a lot of results, so briefly presenting the motivation and reason to study Lie Algebras, limiting ourselves to a special subset of Lie Groups that are specially useful for applications, this approach is found in Brian C. Hall\cite{brian} and a more geometrical approach to the presentation of Lie Algebras with a focus on general Lie Groups can be found in Kirillov\cite{kirillov} and a more direct and algebraic approach can be found in Victor Kac \cite{kaclec}.\\
Lie Groups are sets with a manifold and a group structure, we will focus our attention on matrix Lie Groups and we will define them differently to simplify the extension to Lie Algebras.\\
Given a field $\F=\mathbb{C}$ or $\mathbb{R}$, the general linear group denoted $GL(n;\F)$ is the group of $n\times n$ invertible matrices, a sequence of matrices $A_n$ is gonna converge to $A$ in that space if it converges entry wise $a^n_{ij} \rightarrow a^n$, possible because of the topological structure of the fields considered, this is known as the $\infty$-norm and it is equivalent to the euclidean one.
\begin{defi}
	A matrix Lie Group $G$ is a closed subgroup of $GL(n;\F)$, meaning that it is a group and any sequence of matrices in $A_m \in G$ that converges to a matrix $A \in GL(n;\F)$, then $A \in G$.
	\label{MATRIXLIEGROUPDEF}
\end{defi}
\begin{ex}
	$GL(n;\F)$ is a Matrix Lie Group since it is a group and the closure is trivial.
\end{ex}
\begin{ex}
Since the determinant is a continuous function, then the group $SL(n;\F)$ of matrices with determinant $1$ is a matrix Lie Group
\end{ex}
\begin{ex}A subgroup of $GL(2;\mathbb{C})$ that is not closed, given $a\in \mathbb{R}/\mathbb{Q}$ irrational then:
		$$G=\left\{\left.\begin{pmatrix}
		e^{it} & 0 \\
		0 & e^{ita}	\end{pmatrix} \right| t \in \mathbb{R}\right\}.$$
		Since $-I$ is not in $G$ since $t$ has to be an odd multiple of $\pi$ in which case $ta$ cannot be an odd multiple of $\pi$ since $a$ is irrational, on the other hand we can take $t=(2n+1)\pi$ for carefully chosen integers to make $ta$ arbitrarily close to an odd multiple of $\pi$.\\
\end{ex}
	We will briefly remind some of the properties of matrix exponentials necessary to demonstrate properties of the Lie Algebra as the "logarithm" of a Lie Group in some sense.
\begin{ex}
	ADD MORE EXAMPLES \label{11EXAMPLES}
\end{ex}
\begin{prop}[Matrix Exponential]
	The matrix exponential of a complex or real $X \in M_{n\times n}$ defined as 
	$$e^X = I + \sum_{m=1}^\infty \frac{X^m}{m!}.$$
	satisfy the following properties:
	\begin{enumerate}[label=(\alph*)]
		\item It is a continuous map on the respective matrix spaces.
		\item $(e^X)^*=e^{X^*}$.
		\item If $XY=YX$ then $e^{X+Y}=e^Xe^Y$. As a consequence, $e^X$ is invertible with inverse $e^{-X}$ since $e^0=I$
		\item If $C$ is any invertible matrix, then $e^{CXC^{-1}}=Ce^XC^{-1}$.
		\item $\|e^X\|\le e^{\|X\|}$ (supremum norm).
		\item $\frac{d}{dt}e^{tX} = Xe^{tX}$
		\item \textbf{(Lie's Product Formula)} $e^{X+Y} = \displaystyle\lim_{m \rightarrow \infty} (e^{\frac{X}{m}}e^{\frac{Y}{m}})^m$.
	\end{enumerate}
	\label{MatrixExponential}
\end{prop}
\begin{proof}
	\begin{enumerate}[label=(\alph*)]
		\item Since the max norm($\|\ \|_\infty$) is equivalent to the supremum one, we will show that the series converges absolutely and then prove it is continuous.\\
		It converges absolutely because:
		\begin{equation}
			\sum_{m=1}^\infty \left\|\frac{X^m}{m!}\right\| = \sum_{m=1}^\infty \frac{\|X^m\|}{m!} \le \sum_{m=1}^\infty \frac{\|X\|^m}{m!} = e^{\|X\|}-1.	
		\end{equation}
		Now given $X,Y \in M_{n\times n}$ it follows that(abusing notation as to say $0_{n\times n}^0=1$):
		\begin{align*}
		\|e^{X+Y}-e^X\| &=\left\|\sum_{m\ge 0}\frac{(X+Y)^m - X^m}{m!}\right\|\\
		&\le \sum_{m\ge 0} \frac{(\|X\|+\|Y\|)^m - \|X\|^m}{m!}\\
		&= e^{\|X\|+\|Y\|}-e^{\|X\|}\\
		&= e^{\|X\|} (e^{\|Y\|}-1) \le \|Y\|e^{\|X\|}e^{\|Y\|}.
		\end{align*}
		Continuity follows directly by choosing $Y$ in the neighborhood of a chosen $X$.
		\item Follows directly from $(X^m)^*=(X^*)^m$ and continuity of the matrix adjoint
		\item Since $XY=YX$ then $(X+Y)^m=\sum_{k=0}^\infty {n\choose k}X^kY^{m-k}$ now since the exponential converges absolutely:
		$$e^Xe^Y = \sum_{m=0}^\infty\sum_{k=0}^m \frac{X^k}{k!}\frac{Y^{m-k}}{(m-k)!}=\sum_{m=0}^\infty \frac{1}{m!}\sum_{k=0}^m{n\choose k}X^kY^{m-k} = \sum_{m=0}^\infty \frac{(X+Y)^m}{m!} = e^{X+Y}.$$
		\item Directly:
		$$e^{CXC^{-1}} = \sum_{m=0}^\infty \frac{(CXC^{-1})^m}{m!} = \sum_{m=0}^\infty \frac{CX^mC^{-1}}{m!} = C e^X C^{-1}.$$
		\item The same result presented in $(1.1)$
		\item Again we differentiate it directly term by term since the power series converges uniformly.
		$$\frac{d}{dt} e^{tX} = \frac{d}{dt}1+\sum_{m=1}^\infty \frac{d}{dt}\frac{(tX)^m}{m!} = \sum_{m\ge 1} \frac{t^{m-1}X^m}{(m-1)!} = Xe^{tX}.  $$
		\item Define $A= e^{(X+Y)/k}$ and $B=e^{X/k}e^{Y/k}$, then by the norm inequality from $(1.1)$ and the triangle inequality imply:
		$$\|A\|,\|B\|\le (e^{\|A\|+\|B\|})^{1/k}.$$
		On the other hand, reorganizing terms for $B$ in terms of the power of $k$ by absolute convergence of the exponential:
		$$B=\sum_{i=0}^\infty \frac{(X/k)^i}{i!}\cdot  \sum_{j=0}^\infty   \frac{(Y/k)^j}{j!}= \sum_{m=0}^\infty k^{-m} \sum_{i=0}^m \frac{A^i}{i!}\cdot \frac{B^{m-i}}{(m-i)!}.$$
		Which allows us bond the norm of the difference by:
		\begin{align*}
		\|A-B\| &= \left\|\sum_{i=0}^\infty \frac{([A+B]/k)^i}{i!} - \sum_{m=0}^\infty k^{-m} \sum_{j=0}^m \frac{A^i}{i!}\frac{B^{m-i}}{(m-i)!}\right\|\\
		&=\left\|\sum_{i=2}^\infty k^{-i}\frac{(A+B)^i}{i!} - \sum_{m=2}^\infty k^{-m}\sum_{j=0}^m \frac{A^i}{i!}\frac{B^{m-i}}{(m-i)!}\right\|\\
		&\le \frac{1}{k^2}\left[e^{\|A\|+\|B\|}+\sum_{m=2}^\infty \frac{1}{m!}\sum_{i=0}^m \frac{m}{i}\|A^i\| \|B^{m-i}\|\right]\\
		&= \frac{1}{k^2}\left[e^{\|A\|+\|B\|} + \sum_{m=2}^\infty \frac{(\|A\|+\|B\|)^m}{m!}\right]\\
		&\le \frac{2}{k^2}e^{\|A\|+\|B\|}.
		\end{align*}
	\end{enumerate}
\end{proof}\\
Now we are ready to present and prove some properties of the Lie Algebra of a matrix Lie Group.
\begin{defi}
	Given a matrix Lie Group $G\subset M_{n\times n}(\F)$ then its Lie Algebra is the set $\g = \{X \in M_{n\times n}(\F)\ |\ e^{tX} \in G \text{ for all } t\in\mathbb{R}\}$
	\label{LIEALGEBRAFROMMLG}
\end{defi}
Considering this set instead of the Lie Group itself is very useful as they have very nice algebraic properties and an underlying structure that is quite rich and unique.\\
One of the nicest properties this set posseses is that it is a vector subspace of $M_{n\times n}$ and is closed under a special operator, in fact:
\begin{prop}
Given any two elements $X,Y$ in a Lie Algebra $\g$ of a Lie matrix group G, then:
\begin{enumerate}[label=(\alph*)]
	\item $sX \in \g \text{ for all } s \in \mathbb{R}$
	\item $X+Y \in \g$
	\item $XY-YX \in \g$
\end{enumerate}
\end{prop}
\label{11LieAlgebraAxiomaticDeduction} 
\begin{proof}
\begin{enumerate}[label=(\alph*)]
	\item $e^{t(sX)}=e^{(ts)X} \in G$ for all $t$ since $ts \in \mathbb{R}$
	\item We will use Lie's product formula(Proposition \ref{MatrixExponential}g):
	$e^{t(X+Y)} = \lim_{m \rightarrow \infty}(e^{tX/m}e^{tY/m})^m$, since $G$ is a group then $(e^{tX/m}e^{tY/m})^m$ is in $G$ for all $m \in \mathbb{N}$ and since it converges therefore by definition its limit is in the Lie Matrix Group $G$, proving that $X+Y \in \g$
	\item As we proved that it is a vector subspace of $M_{n\times n}$, we use the topologically closed property using limits:\\
	Since $e^{tX} \in G$ and $Y \in \g$ then $e^{e^{tX}Ye^{-tX}} = e^{tX}e^{Y}e^{-tX} \in G$ which implies that $e^{tX}Ye^{-tX} \in\g$ and therefore $$\frac{e^{tX}Ye^{-tX}-Y}{t} \in \g.$$ for all $t$, and then its limit as $t\rightarrow 0$ is as well, but 
	$$\lim_{t\rightarrow 0}\frac{e^{tX}Ye^{-tX}-Y}{t} = \frac{d}{dt} \left.e^{tX}Ye^{-tX}\right|_{t=0} = [Xe^{tX}Ye^{-tX} + e^{tX}Y(-X)e^{-tX}]_{t=0} = XY-YX.$$
\end{enumerate}
\end{proof}
\section{Algebras and General Properties}
\begin{defi}
	A pair $(\A,\cdot)$ where $\A$ is a vector space over a field $\F$ and $\cdot:\A\times\A\rightarrow \A$ is bilinear is defined as an $\F$-algebra.
	\label{genalgebra}
\end{defi}
$\F$-algebras are everywhere in mathematical research, from matrix spaces to polynomials and are the main topic of this dissertation.\\
Note that this is a convention used in the studies of linear properties of algebraic structures, and in another context an $F$-algebra can denote something very different.\\
If some additional structure is present in the algebra, we shall denote so by the properties of its product, for example the following:
\begin{itemize}
	\item If for all $x,y \in \A$, $x\cdot y = y \cdot x$ it is called \textbf{commutative}.
	\item If for all $x,y,z \in \A$, $(x\cdot y)\cdot z = x \cdot(y \cdot z)$ it is called \textbf{associative}.
	\item If there exists an element $e \in \A$ such that $e \cdot x = x \cdot e = x$ for all $x \in \A$ then it is called \textbf{unital}.
\end{itemize}
\begin{defi}
A subspace $B$ of an algebra $A$ which satisfies $x,y \in B \Rightarrow x\cdot y \in B$ is called a subalgebra which will be represented by $B\le A$, when it satisfies the further property that $x \in A, y \in B \Rightarrow x\cdot y \in B(y\cdot x\in B)$ it will be called a left-ideal(right-ideal), being simply called an ideal if it satisfies both, which will be denoted by $A\ideal B$ in that case.\\
An algebra morphism is defined as a linear transformation that preserves the product, that is, given two algebras $(A,\cdot),(B,\times)$ then a linear transformation $\varphi:A\rightarrow B$ is such that:
	$$\varphi(x \cdot y) = \varphi(x) \times \varphi(y).$$
\label{12subalgebra}
\end{defi}
It is generally useful on proving properties of algebras to consider different compositions of it, such as a quotient or direct sum, there is a natural way to expand the product by composing algebras:
\begin{prop}
If $B$ is an ideal of $(A,\cdot)$ then the quotient space $A/B$ with the product $\times$ defined as $[x] \times [y] =  [x\cdot y]$ is an algebra and the projection $[\ ]:A\rightarrow A/B$ is an algebra morphism.\\
If $(A,\cdot)$ and $(B,\times)$ are algebras, then the vector space $A\oplus B$ is an algebra with\\ $(a,b)(a',b')=(a\cdot a',b \times b')$
\label{genquotient}
\end{prop}
\begin{proof}
	It just needs to be shown that the product is well defined, but given $x+B$ and $y+B$ then the product $(x+B)\cdot(y+B)=x\cdot y + x\cdot B + B\cdot y + B\cdot B = x\cdot y+B$ is in the same coset independently of the representing vector used. As for the direct sum, bilinearity follows directly from the vector space structure from $A\oplus B$ and the bilinearity of the product.
\end{proof}\\
We will state some theorems that are gonna be useful later for the specific cases of Lie algebras but are valid in the general case of algebras:\\
\begin{prop}
Given algebras $(A,\cdot)$ and $(B,\times)$ and a morphism $\varphi:A\rightarrow B$ then: 
\begin{enumerate}[label=\alph*.]
	\item $\ker(\varphi) \ideal A$
	\item $\text{im}(\varphi) \le B$ 
	\item $A/\ker(\varphi) \simeq \text{im}(\varphi)$ where $(\simeq)$ denotes the existence of a bijective morphism (isomorphism).
\end{enumerate}	
\label{12isomorphism}
\end{prop}
\begin{proof}
	Given $X \in \ker(\varphi)$ and $Y \in A$ then $\varphi(X \cdot Y) = \varphi(X)\times \varphi(Y) = 0$ and therefore $X\cdot Y \in A$ it is an ideal.\\
	Given $\varphi(X),\varphi(Y) \in \text{im}(\varphi)$ then $\varphi(X) \times \varphi(Y) = \varphi(X\cdot Y)$, and since $X \cdot Y \in A$ then $\varphi(X),\varphi(Y) \in \text{im}(\varphi)$\\
	Finally consider the morphism $\psi: X+\ker(\varphi) \mapsto \varphi(X)$, $\psi$ is well defined since $$\psi(X+\ker(\varphi))=\psi(Y+\ker(\varphi)) \Rightarrow \varphi(X)=\varphi(Y) \Rightarrow \varphi(X-Y)=0 \Rightarrow X-Y \in \ker\varphi.$$
	It is a morphism by the way we define the product in the quotient: $$\psi([X]\cdot[Y]) = \psi([X \cdot Y]) = \varphi(X \cdot Y) = \varphi(X)\times \varphi(Y) = \psi([X])\times \psi([Y]).$$
	Finally, it is surjective because if $\psi([X])=0$ then $\varphi(X)=0$ which implies that $[X]=[0]$ and for all $\varphi(X) \in \text{im}(\varphi)$ then $\varphi(X) = \psi([X])$, with $[X] \in A/\ker(\varphi)$
\end{proof}\\
\subsection*{Linear Algebra}
\begin{prop}[Jordan-Chevalley Decomposition]
	If $V$ is a finite dimensional vector space over $\F$ algebraically closed, then every $T:V\rightarrow V$ linear satisfies:
	\begin{enumerate}[label=\Alph*]
		\item There exists unique $S,N:V\rightarrow V$ satisfying $T=S+N$ with $S$ semi-simple and $N$ nilpotent
		\item There exist polynomials $p(x)$ and $q(x)$ in one indeterminate, without constant term such that $S=p(T)$ and $N=q(T)$, in particular, $S$ and $N$ commute with every endomorphism commuting with $T$.
		\item If $A \subset B \subset V$ are subspaces, and $T$ maps $B$ into $A$, then $S$ and $N$ also map $B$ into $A$
	\end{enumerate}
	\label{jordandecom}
\end{prop}
	\begin{proof}
		Let $\Pi(x-a_i)^{m_i}$  be the characteristic polynomial of $T$, then $V=\oplus \ker(T-a_iI)^{m_i}$, by the Chinese Remainder Theorem, we can find a polynomial $p(x)$ such that:$$\begin{cases}
		p(x) \equiv a_i \pmod{(x-a_i)^{m_i}},\\
		p(x) \equiv 0 \pmod x.
		\end{cases}$$
		Now let $q(x)=x-p(x)$ and set $S=p(T)$ and $N=q(T)$, since they are both polynomials in $T$ then they commute with every endomorphism commuting with $T$ and furthermore they send subsets into subsets in accordance with $C.$, to show that $S$ is semi-simple, consider that $S-a_iI$ restricted to $\ker(T-a_iI)^{m_i}$ is $0$ and therefore it acts diagonally on each of those spaces, but then it acts diagonally on $V$ as a direct sum. Now $N$ is clearly nilpotent as it has no non-zero eigenvalues.\\
		To prove uniqueness, consider that $S_1 + N_1 = S_2 + N_2 = T$, then $S_1 - S_2 = N_2 - N_1$ and since sum of commuting semi-simple operators is semi-simple and the same is valid for nilpotent then $S_1-S_2$ is both semi-simple and nilpotent and therefore it is $0$, implying $S_1 = S_2$ and therefore $N_1=N_2$.
	\end{proof}

\section{Lie Algebras: Basics}
Inspired by the properties satisfied by Lie Algebras of matrix groups, we will define general Lie Algebras by those properties.
\begin{defi}
A Lie Algebra is an $\F$-algebra $\g$ with a product (represented by $[\ ,\ ]$) satisfying, for all $X, Y \text{ and }Z \in \g$:
\begin{enumerate}[label=(\alph*)]
	\item $[X,X]=0$ (Anti-Commutativity)
	\item $[X,[Y,Z]]+[Y,[Z,X]]+[Z,[X,Y]]=0$ (Jacobi Identity)
\end{enumerate}
\label{13LieAlgebra}
\end{defi}
In the case of matrices, we generally define $[X,Y]=XY-YX$ and the properties are easily verified:
\begin{enumerate}[label=(\alph*)]
	\item $[X,X]=X^2-X^2 = 0$
	\item $[X,[Y,Z]]+[Y,[Z,X]]+[Z,[X,Y]]=(XYZ - XZY - YZX + ZYX) + (YZX - YXZ - ZXY + XZY) + (ZXY - ZYX-XYZ+YXZ)=0$
\end{enumerate}
and therefore we prove that the set of all $n\times n$ matrices, in fact over any field is a Lie Algebra with the commutator, there is nothing special about finite dimensions or a specific base here, and if we extend this to endomorphisms on a vector space $V$ over $\F$ we can denote the Lie Algebra by $\gl(V)$ or denoted by $\gl(n,\F)$ with matrices, in respect to the basis $\{e_{ij}\}, 1\le i,j \le n$ the Lie Bracket is given directly by: $[e_{ij},e_{kl}] = \delta_{jk}e_{il} - \delta_{il}e_{kj}$ in the finite dimensional case.
\begin{ex}
	Matrix Algebras
	\begin{enumerate}
		\item $\gsl(n,\F) = \{X \in \gl(n,\F)|\Tr(X)=0\}$, since $\Tr:\gl \rightarrow \F$ is a linear form then $\gsl$ is a vector space as the kernel, furthermore it is closed under the Lie Bracket, since $\Tr(XY)=\Tr(YX)$
		\item If $B:\F^n \times \F^n \rightarrow \F$ is a bilinear form then the subspace: $$\mathfrak{o}_B(n,\F) = \{X \in \gl(n,\F)| B(Xv,w)+B(v,Xw)=0 \text{ for all } v,w\in \F^n\}$$ is a Lie Algebra.\\
		It is easily seen as a subspace because of bilinearity, now to check closure under the bracket:
		\begin{align*}
		B((XY-YX)v,w) &= B(XYv,w) - B(YXv,w) = -B(Yv,Xw) + B(Xv,Yw)\\
		&= B(v,YXw)- B(v,XYw) = -B(v,(XY-YX)w).
		\end{align*}
		\item The upper triangular and strictly upper triangular matrices are a Lie Algebra, in fact if the matrices are $\sum_{i<j} a_{ij} e_{ij}$ then since $[a_{ij}e_{ij},b_{kl}e_{kl}] = a_{ij}b_{kl}(\delta_{jk}e_{il}-\delta_{il}e_{kj})$
		if $j=k$ then $i<j=k<l$ and if $i=l$ then $k<l=i<j$, the same is valid for the space generated by $\{e_{ij},i\le j\}$
	\end{enumerate}
\end{ex}
\begin{ex}
	MORE EXAMPLES HERE TODO \label{MOREEXAMPLES}
\end{ex}
\begin{defi}
	Given a Lie Algebra $\g$ a derivation of $\g$ is a linear transformation $D\in\gl(\g)$ such that:
	$$D[X,Y]= [DX,Y]+[X,DY]$$
	The adjoint of an element $X \in \g$ is defined as $\ad(X) \in \gl(\g)$ in such a way that $\ad(X)(Y) = [X,Y]$\\
	A representation of $\g$ in a vector space $V$ is a Lie Algebra morphism $\rho$ from $\g$ to $\gl(V)$, meaning:
	$$\rho([X,Y]) = \rho(X)\rho(Y)-\rho(Y)\rho(X)$$
	The center of a Lie Algebra defined as $\z(\g) = \{Z \in \g | [Z,X] = 0 \text{ for all } X \in \g\}$ is an ideal.
\end{defi}
\begin{prop}
	The adjoint of an element is a derivation and the adjoint representation $\ad: X \mapsto \ad(X)$ is a representation from $\g$ into itself.
	\label{13adjointrepresentation}
\end{prop}
\begin{proof}
	Using the Jacobi identity:
	\begin{align*}
	\ad(X)([Y,Z]) &= [X,[Y,Z]] = -[Y,[Z,X]]-[Z,[X,Y]]\\
	&= [Y,[X,Z]] + [[X,Y],Z] = [Y,\ad(X)Z] + [\ad(X)Y,Z].
	\end{align*}
	\begin{align*}
	\ad([X,Y])(Z) &= [[X,Y],Z] = [X,[Y,Z]] + [Y,[Z,X]] \\
	&= [X,[Y,Z]] - [Y,[X,Z]] = \ad(X)\ad(Y)Z - \ad(Y)\ad(X)Z \ \text{ for all } Z \in \g.
	\end{align*}
\end{proof}\\
A useful way to think of representations is thinking them as linear actions that somehow preserve the structure of the Lie Algebra so we will choose a more direct way to define them.
\begin{defi}
	A $\g$-module is a vector space $V$ together with an operator $\cdot: \g \times V \rightarrow V$ in such a way that:
	\begin{enumerate}[label=(\alph*)]
		\item $X \cdot (v + w) = X \cdot v + X \cdot w$
		\item $(X+Y) \cdot v = X \cdot v + Y \cdot v$
		\item $[X,Y] \cdot v = X \cdot (Y \cdot v) - Y\cdot (X \cdot v)$
	\end{enumerate}
	\label{gmod}
\end{defi}
\begin{prop}
	Every $\g$-module $(V,\cdot)$ defines a representation and every representation $\rho$ defines a $\g$-module in such a way that $a \cdot v = \rho(a)v$ for every $a \in \g$ and $v \in V$
	\label{modequivrep}
\end{prop}
\begin{proof}
	The result is trivial, but for completeness:\\
	Given $V$ a $\g$-mod, then define $\rho(a) \in \gl(V)$ such that $\rho(a): v \mapsto a \cdot v$, it is in $\gl(V)$ if and only if the action satisfies $(a)$, it is a q linear morphism if and only if $(b)$ and it satisfies $\rho([X,Y])=\rho(X)\rho(Y)-\rho(Y)\rho(X)$ if and only if $(c)$
\end{proof}\\
Now we will move on to the first important results of Lie Theory, and for that we will define two series of ideals that play an important role. But before that, some clarification on notation:\\
Given subsets $\h$ and $k$ of a Lie Algebra $\g$ , we will define $[\h,k]$ as the set of all products from $\h$ and $k$, in case $\h$ and $k$ are vector spaces the order is irrelevant because of skew-symmetry and the resulting set is going to be a vector space by the bilinearity of the Lie Bracket . $$[\h,k] = \{[H,K] \in \g | H \in \h, K \in k\}$$
\begin{defi}
	The descendant series defined recurrently as:$$\begin{cases}
	\g^1 = \g, \\
	\g^n = [\g,\g^{n-1}].
	\end{cases}$$.\\
	The derived series is also defined recurrently: $$\begin{cases}
	\g^{(1)} = \g,\\
	\g^{(n)} = [\g^{(n-1)},\g^{(n-1)}].
	\end{cases}$$
	\label{13series}
\end{defi}
Some properties of these series are directly proven:
\begin{prop}
	\begin{enumerate}[label=\alph*.]
		\item They are decrescent series with respect to inclusion, that is:\\
		$\g \supset \g^2 \supset \cdots \supset \g^n \supset \cdots $
		and similarly $\g \supset \g^{(2)} \supset \cdots \supset \g^{(n)} \supset \cdots$
		\item All members of the series are ideals of $\g$
		\item For all $n$,  $\g^{(n)}\subset \g^n$
	\end{enumerate}
	\label{13seriesprop}
\end{prop}
\begin{proof}
	\begin{enumerate}[label=(\alph*)]
		\item For the case of the descendental series, consider an element in $X \in \g^n$ for $n > 2$ the case when $n=2$ follows directly from $\g$ being a Lie Algebra, by induction assume $g^{n}\subset g^{n-1}$ and then $X \in \g^{n+1}$ and then: $X=[Y,Z]$ for $Z \in \g^{n}$ and $Y \in \g$ then by induction $Y\in \g^{n-1}$ and we are done since $X \in [\g^{n-1},\g] = g^{n}$.\\
		For the case of the derived algebra, we also proceed by induction, the case $g^{(2)} = [g,g] \subset g$ is direct, assuming $g^{(n)} \subset g^{(n+1)}$ for $X \in g^{(n+1)}$ with $X=[Y,Z]$ for $Y,Z \in g^{(n)}$, then since $Y,Z \in g^{(n-1)}$ the result follows directly.\\
	\item For the descendental series, the result is direct from the previous result, $[g,g^n] = g^{n+1} \subset g^n$, it being a vector space follows from the previous discussion about the bracket of sets. For the derived series, we will use induction since $\g^{(1)}=\g \ideal \g$ trivially, then assume that $\g^{(n)} \ideal \g$ and let $X \in \g$ and $Y \in \g^{(n+1)}$ since $Y$ can be written as $[Z,W]$ for some $Z,W \in \g^{(n)}$, then:
	$$[X,Y] = [X,[Z,W]] = -[Z,[W,X]]-[W,[X,Z]] $$ 
	by induction $[W,X] = -[X,W] \in \g^{(n)}$ and therefore $[X,Y]$ is the sum of two elements in $\g^{(n+1)}$, given that it is a vector space then $[X,Y]$ is in $\g^{(n+1)}$, proving it is an ideal.
	\item By induction since $\g^{(n)}\subset \g$ then $\g^{(n+1)} = [\g^{(n)},\g^{(n)}] \subset [\g,\g^n] = \g^{n+1}$.
	\end{enumerate}
\end{proof}\\
Some type of algebras are classically special in the theory, with the ideal series being defined, then:
\begin{defi}

 An algebra is called \textbf{nilpotent} if $\g^n = 0$ for some $n$.\\
 An algebra is called \textbf{solvable} if $\g^{(n)}=0$ for some $n$.\\
 An algebra is \textbf{semi-simple} if it has no non-zero proper solvable ideals.\\
 An algebra is \textbf{simple} if it has no non-zero proper ideals.

\label{algebratypes}
\end{defi}

\section{Nilpotent and Solvable Algebras}
This first section of results focuses on results about finite dimensional nilpotent algebras first focusing on matrix algebras, the results here are self-contained and play an important role in representation theory.\\
We first correlate nilpotent matrices to ad-nilpotent matrices, the following proposition is a pre requisite to all further proves in this section:
\begin{prop}
If $X \in \gl(V)$ is nilpotent with $X^n=0$ then $\ad(X)$ is nilpotent, that is, for some $N$ $\ad(X)^NY = 0$ for all $Y \in \gl(V)$.
\label{propadnilpotency}
\end{prop}
\begin{proof}
Let $L_X$ and $R_X$ in $\gl(\gl(V))$ be the left and right multiplication by $X$ respectively, then since $L_X^n = R_X^n = 0$, since $\ad(X) = L_X - R_X$ and $L_XR_XY = XYX = R_XL_XY$, we just use the binomial theorem for commuting operators:
$$\ad(X)^{2n} = (L_X-R_X)^{2n} = \sum_{k=0}^{2n} (-1)^{2n-k}{2n \choose k} L_X^{k} R_X^{2n-k}.$$
Which is $0$ term by term because $R_x^{2n-k}=0$ for $k<n$ since $2n-k>n$, and if $k\ge n$ then $L_x^k=0$.
\end{proof}
\begin{teo}[Engel's Lemma]
	 Let $\g$ be a sub algebra of $\gl(V)$, if $\g$ consists of nilpotent endomorphisms and $V\not=0$, then there exists a nonzero vector $v \in V$ for which $Xv=0$ for all $X \in v$.
	 \label{Engel's Lemma}
\end{teo}
\begin{proof}
	We will proceed by induction, the case when $\dim \g = 1$ with $\{X\}$ as a basis is satisfied by taking any vector $v\not=0$ and considering the smallest value of $n$ such that $X^nv=0$, then $X(X^{n-1}v) = 0$ and $X^{n-1}v \not= 0$.\\
	If $\h$ is any proper and nonzero subalgebra of $\g$, then since $\ad(H)$ is nilpotent for any $H \in \h$ then it also acts nilpotently on the quotient space $\g/\h$ with well defined action since it is a subalgebra, so by the induction hypothesis there exists some $\overline{X_0} \in \g/\h$ such that $[H,\overline{X_0}] \in \h$ therefore since $[X_0,X_0]=0$ then $\h \oplus \F X_0$ is a subalgebra of $\g$.\\
	Assuming $\dim \g\ge2$ then there exists a starting $\h$ as above because any non-zero element of $\g$ forms a subalgebra, now assume  $\h$ to be a maximal proper subalgebra with respect to inclusion then since $\h + \F X_0$ as above is a subalgebra that includes $\h$, by maximality $\g = \h + \F X_0$ and $\h$ is an ideal because $[\g,\h] = [\h,\h]+ [\F X_0,\h] \subset \h.$\\
	Also by the induction, $W=\{v \in V|Hv = 0 \text{ for all } H \in \h\}$ is non-zero, but since $\h$ is an ideal, this space is stable under $\g$, let $X \in \g$ ,$Y \in \h$ and $w \in W$:
	$YXw = XYw - [X,Y]w = 0$
	Finally $X_0$ acts on $W$ and is nilpotent, implying there exists a non-zero vector $v \in W$ such that $X_0v=0$, therefore $X v = 0$ for any $X \in \g$.
\end{proof}\\
\begin{teo}[Engel's Theorem]
If all elements of $\g$ are ad-nilpotent, then $\g$ is nilpotent.
\label{Engel's Theorem}
\end{teo}
\begin{proof}
The image of the adjoint representation satisfies the conditions of Theorem \ref{Engel's Lemma} in $\gl(\g)$, implying that there exists non-zero $X \in \g$ such that $[X,\g]=0$, therefore $\z(\g)\not=0$, now $\g/Z(\g)$ consists of ad-nilpotent and has smaller dimension than $L$, but if $\g/\z(\g)$ is nilpotent then $\g$ is also nilpotent, in fact if $\g/(\z(\g))$ is nilpotent then $\g^n \subset \z(\g)$ for some $n$, but that implies $\g^{n+1} \subset [\g,\z(\g)] = 0$.
\end{proof}\\
A similar result is valid for solvable Lie Algebras under other conditions, consider $\F$ to have characteristic $0$ and be algebraically closed, then the following is valid:
\begin{teo}[Lie's Theorem]
	Let $\g$ be a solvable Lie Algebra of $\gl(V)$, $V$ finite dimensional, then $V$ contains a vector $v$ and a linear $\lambda:\g \rightarrow \F$ such that $Xv = \lambda(X)v$ for all $X \in \g$.
	\label{Lie's Theorem}
\end{teo}
\begin{proof}
	The proof of this theorem follows the same steps as for Engel's Lemma (Theorem \ref{Engel's Lemma}).
	\begin{enumerate}
		\item Locate an ideal $\h$ of $\g$ with co-dimension one.
		\item Common eigenvector exist for $\h$ by induction.
		\item $\g$ stabilizes the space consisting of these eigenvectors.
		\item Find an eigenvector in that space for a single $X_0 \in \g$ such that $\g = \h \oplus \F X_0$.
	\end{enumerate}
	\begin{enumerate}
		\item Since $\g$ is solvable then $\g^{(2)}=[\g,\g]\not=\g$, therefore any co-dimension one vector space  containing $\g^{(2)}$ is an ideal.\\
		\item Proceeding by induction on $\dim \g$, if $\dim \g=0$ then the result follows by vacuity, therefore since $\dim \h = \dim \g - 1 < \dim \g$ then we assume the existence of a nonzero common eigenspace. $$W = \{v \in V | Xv = \lambda(X)v \text{ for some } \lambda \in \h^* \text{and for all }X \in \h\}.$$
		\item To prove that $\g$ stabilizes $W$, let $w \in W$, $X \in \g$ and $Y \in \h$ arbitrary. Then $YXw = XYw - [x,y]w = \lambda(Y)Xw - \lambda([X,Y])w$. Thus proving $\lambda([X,Y])=0$ will prove our desired result.\\
		For this let $n$ be the smallest integer such that $\{w,Xw,\cdots,X^nw\}$ is linearly independent, and let $V_i = \langle w, Xw , \cdots, X^i w\rangle$ then $V_{N} = V_n$ for $N>n$ and $\dim V_n = n$, $\h$ stabilizes each $V_i$ which can easily be proven by induction:
		$$YXX^{n-1}w = XYX^{n-1}w - [X,Y]X^{n-1} = (XY - [X,Y])X^{n-1}w$$ since $X$ clearly stabilizes $V_i$ and $Yw = \lambda(Y)w$.\\
		Now from this we know that $\Tr_{V_n}(Y)=n\lambda(Y)$ but this is valid for the special element in $Y$ of the form $[X,Y]$, since $\Tr([X,Y])= \Tr(XY)-\Tr(YX)=0$ then $n\lambda([X,Y])=0$. Since we assume characteristic $0$ this forces $\lambda[X,Y]=0$.
		\item Since $X_0: W \rightarrow W$, then there exists an eigenvector $v \in W$ for $X_0$, just extend $\lambda: \h \rightarrow \F$ to include $X_0$ in its domain and $\lambda(X_0)$ as the eigenvalue.
		\end{enumerate}
\end{proof}
\section{Cartan's Criteria}
Here we will consider a finite dimensional vector space $V$ over an algebraically closed field $\F$.
\begin{prop}
	If $X \in \gl(V)$ is such that its Jordan decomposition(Proposition \ref{jordandecom}) is $X=S+N$ then $\ad X = \ad S + \ad N$ is the Jordan decomposition of $\ad X$ in $\gl(\gl(V))$.
\end{prop}
\begin{proof}
	A basis of $\gl(V)$ is $e_{ij}$ with respect to a basis such that $S= \text{diag}{\lambda_i}$ then $[S,e_{ij}]=(\lambda_i-\lambda_j)e_{ij}$, therefore $\ad(S)$ is semi-simple, if $N$ is nilpotent then $\ad(N)$ is nilpotent (Proposition \ref{propadnilpotency}). Since the adjoint is a representation, then $[\ad S,\ad N]_{\gl(\gl(V))} = \ad[S,N] = 0$ and then the result follows from the uniqueness of Jordan Decomposition in $\gl(\gl(V))$.
\end{proof}
\begin{teo}[Cartan's Lemma]
	Let $A \subset B$ be two subspaces of $\gl(V)$ with $\dim V < \infty$ over a field with characteristic $0$, then set $M=\{X \in \gl(V)\ |\ [X,B]\subset A\}$. If $X \in M$ is such that $\Tr(XY)=0$ for all $Y \in M$. Then $X$ is nilpotent.
	\label{Cartan's Lemma}
\end{teo}
\begin{proof}
	This proof consists mostly of technicalities on proving that specific matrices are present in $M$.\\ 
	Let $X=S+N$ be the Jordan decomposition of $X$, fix a basis $\{v_1,\cdots,v_m\}$ in which $S=\text{diag}(a_1,\cdots,a_m)$. We want to prove that $a_1=\cdots=a_m=0$ since in that case $S=0$ and $X$ is nilpotent under the hypothesis.\\
	Let $E$ be the subspace of $\F$ generated by the $a_i$ over $\mathbb{Q}$ since we are assuming $\text{char } \F= 0$, then it is enough to show that $E=0$, or equivalently $E^*=0$.\\
	Let $f$ be a linear functional, and $Y=\text{diag}(f(a_1),\cdots,f(a_m))$. If $\{e_{ij}\}$ is the corresponding basis of $\gl(V)$, then $\ad(S)(e_{ij})` = (a_i-a_j)e_{ij}$ and $\ad(Y)(e_{ij}) = (f(a_i)-f(a_j))e_{ij}$. Now let $r(x)\in\F[X]$ be the polynomial such that $r(a_i-a_j)=f(a_i)-f(a_j)$, the existence of which follows from the Lagrange polynomial and linearity. Then $\ad Y = r(\ad S)$.\\
	Now $\ad S$ is the semi-simple part of $\ad X$, so it can be written as a polynomial in $\ad X$ without constant term by (Proposition `\ref{jordandecom}B.) and therefore $\ad Y$ as a polynomial in $S$ maps $B$ to $A$, proving that $Y$ is in $M$.\\
	Now $\Tr(XY)=\displaystyle\sum_{j=0}^m a_jf(a_j)=0 \Rightarrow \displaystyle\sum_{j=0}^m f(a_j)^2 =0$ by applying $f$ to the equality which in turn, since we restricted ourselves to $\mathbb{Q}$, implies that $f(a_j)=0$ for all $j$ and therefore $f=0$. Since $f$ is arbitrary then $E^*=0$.
\end{proof}	\\
This Lemma, very technical and apparently not very useful is essential to one of the most important facts about Lie Algebras as means of characterization of semi-simple Lie Algebras, in fact:
\begin{teo}[Cartan's Criterion]
	Let $\g$ be a subalgebra of $\gl(V)$, $V$ finite dimensional over a field $\F$ with characteristic $0$. Suppose that $\Tr(XY)=0$ for all $X \in [\g,\g]$ and $Y \in \g$. Then $\g$ is solvable.
	\label{Cartan's Criterion}
\end{teo}
\begin{proof}
	Let $A=[\g,\g]$ and $B=\g$, then the hypothesis shows that $\Tr(XY)=0$ for all $X \in A$ and $Y \in B$. We need a stronger statement to use the lemma, which is that for all $X \in A$ and $Y \in M$ it follows that $\Tr(XY)=0$, where $M=\{X \in \gl(V)| [X,B]\subset A\}$.\\
	Consider that $\Tr([X,Y]Z)=\Tr(X[Y,Z])$ for all $X,Y,Z \in \gl(V)$, since $$\Tr(XYZ - YXZ) = \Tr(XYZ)-\Tr(YXZ) = \Tr(XYZ)-\Tr(Y(XZ)) = \Tr(XYZ-XZY) = \Tr(X[Y,Z])$$
	Then let $[X,Y] \in A$ with $X$ and $Y \in \g$ and $Z \in M$, then:$\Tr([X,Y]Z)=\Tr(X[Y,Z])=0$ since $[Y,Z] \in A$ and $X \in B$. Therefore by Theorem \ref{Cartan's Lemma} every element in $[\g,\g]$ is nilpotent, which implies that $[\g,\g]$ is a nilpotent algebra, and therefore since $\g^{(n)}\subset \g^n = [\g,\g]^{n-1}=0$ for some $n$, $\g$ is solvable.
\end{proof}\\
This in turn allows us to classify algebras with respect to the trace form in the adjoint, in fact, let:
\begin{corol}
	Let $\g$ be any finite dimensional Lie Algebra over a field $\F$ algebraically closed with characteristic $0$, then if $\Tr(\ad(X) \ad(Y))=0$ for all $X \in [\g,\g]$ and $Y \in \g$. Then $\g$ is solvable.
	\label{Corollary}
\end{corol}
\begin{proof}
	We prove that $\ad \g = \g/\ker \ad$ is solvable from the theorem above, since $\ker \ad = \z(g)$ is solvable then $\g$ is solvable.\\
	It only remains to prove the following lemma: If $\h$ is a solvable ideal of $\g$ and $\g/\h$ is solvable, then $\g$ is solvable. To prove that, consider that $\h^{(n)} = 0$ for some $n$ and $(\g/\h)^{(m)} = [0]$ for some $m$, then $(\g^{(m)})^{(n)} \subset \h^{(n)} = 0$
\end{proof}\\
With this in mind, define the natural bilinear form in Finite Dimensional Lie Algebras, the Trace Form or Cartan-Killing form.
\begin{defi}
	Let $\g$ be any Lie Algebra. Define $\kappa(X,Y)= \Tr [\ad(X)\ad(Y)]$ in $\gl(\g)$ for $X$ and $Y$ in $\g$, then $\kappa$ is a symmetric bilinear form of $\g$, called the Killing form and is also invariant in the sense that $\kappa([X,Y],Z)=\kappa(X,[Y,Z])$.\\
	Its radical, called $\Rad \g = \{X \in \g | \kappa(X,Y)=0 \text{ for all } Y \in \g\}$ is an ideal of $\g$. 
	\label{Killing Form}
\end{defi}
With this in mind we can make the first step to classify semi-simple finite dimensional Lie Algebras.
\begin{teo}
	Let $\g$ be a finite dimensional Lie Algebra over $\F$ algebraic closed and over a field of characteristic $0$. Then the following are equivalent:
	\begin{enumerate}
		\item $\g$ is semi-simple.
		\item $\g$ has no non-zero abelian ideals.
		\item The Killing Form $\kappa(X,Y)=\Tr(\ad(X)\ad(Y))$ is non-degenerate
		\item $\g$ is a unique sum of simple ideals $\g_i$.
	\end{enumerate}
\end{teo}
\begin{proof}\\
	$1 \Rightarrow 2$, to prove this we will use a simple lemma, let $\h \ideal \g$, then $\h^{(k)}$ is an ideal of $\g$ for all $k$:\\
	By induction, basis case $k=1$ is the hypothesis, let $[X,Y] \in \h^{n}$, with $X,Y \in \h^{(n-1)}$ and $Z \in \g$, then:
	$$[Z,[X,Y]] = -[X,[Y,Z]]-[Y,[Z,X]] = [[Y,Z],X] + [[Z,X],Y]$$
	and by induction $[Y,Z],[Z,X]\in \h^{(n-1)}$ therefore $[Z,[X,Y]] \in [\h^{(n-1)},\h^{n-1}] = \h^{(n)}$.\\
	Finally if $\h$ is solvable with $\h^{(n)}=0$, then $\h^{(n-1)}$ is abelian, and by the previous discussion, it is an abelian ideal of $\g$.\\
	$2 \Rightarrow 3$, we will prove a more general result, that any abelian ideal contains the radical of the Killing form. In fact let $\h$ be an abelian ideal, then for all $X \in \h$ and $Y \in \g$ we get $\ad X \ad Y : \g \rightarrow \h$, which in turn implies that $(\ad X \ad Y)^2 : \g \rightarrow [\h,\h] = 0$, and then the operator $\ad X \ad Y$ is nilpotent, which implies that it has trace $0$, proving that $\kappa(X,Y)=0$ for all $X \in \h$ and $Y \in \g$, then by Cartan's lemma $\h$ is solvable and therefore if the Killing form has a non-zero radical then it has an abelian ideal.\\
	$3 \Rightarrow 4$, since $\kappa$ is a non-degenerate bilinear form, given an ideal $\h$ let $$\h^\perp = \{X \in \g | \kappa(X,Y)=0 \text{ for all } Y \in \h\}$$ we get $\g = \h \oplus \h^\perp$, $\h^\perp$ being an ideal of $\g$ by invariance, in fact given $X\in \h^\perp$ and $Z \in \g$ then for all $Y \in \h$ we get: $\kappa([X,Z],Y) = \kappa(X,[Z,Y])=0$ since $\h$ is an ideal.\\
	With this, we proceed the proof of existence. If $\g$ is simple, then $\g$ is the only ideal, and therefore it's done, otherwise, $\g$ has an ideal $\h$, and therefore $\g = \h \oplus \h^\perp$, $\h^\perp$ being a non-zero ideal of $\g$, the result follows by doing the same argument for $\h$ and $\h^\perp$ and from the fact that $\g$ is finite-dimensional.\\
	To prove uniqueness, assume the existence of a decomposition $\g = \bigoplus \g_i$, and let $\h$ be any simple ideal of $\g$, then from the fact that $\h$ is an ideal $[\g,\h]$ is an ideal contained in $\h$ and it is nonzero because otherwise $\h\subset \z(\g)=0$, and since $\h$ is simple then $[\g,\h]=\h$. On the other hand $[\g,\h]=\bigoplus[\g_i,\h]$, but since $[\g,\h]=\h$ is simple, then all terms are $0$ except one, proving that $\h=\g_j$ for some $j$.\\
	$4 \Rightarrow 1$ Let $\g= \bigoplus_{i \in I} \g_i$ be the decomposition of $\g$ in simple ideals, then let $\h$ be any ideal of $\g$, therefore $[\g,\h] = \bigoplus_{i \in I} [\g_i,\h]$, but for any specific $i$, then $[\g_i,\h]\ideal \g_i$, therefore it's either $0$ or $\g_i$, implying that $\h$ is a sum of $\g_i$, when $i$ is running on a subset of $I$, the result follows from induction on the size of the decomposition, since when $\g$ is simple, then the ideal can't be solvable.
\end{proof}
\section{Weyl's Theorem}
\subsection*{Modules}
Before we start talking about the Theorem itself, we remind the notion of representations presented before and using the notation of modules we will talk about composing representations in the most natural way, and using those composition we are able to prove the Weyl's Theorem of irreducibility.\\
We remind the three properties that define a module, $V$ is a vector space and $\g$ is a Lie Algebra, with $v,w \in V$ and $X,Y \in \g$, and an action from $\g$ to $V$ represented by concatenation.
\begin{enumerate}[label=\Alph*.]
	\item $X(v+w) = Xv+Xw$ 
	\item $(X+Y)v = Xv + Yv$
	\item $[X,Y]v = X(Yv) - Y(Xv)$
\end{enumerate}
Our first results focuses on the composition of different modules with respect to their vector space structure
\begin{prop}
	Let $V$ and $W$ be $\g$-modules, and $X \in \g$ a generic element, then:
	\begin{enumerate}
		\item $V \oplus W$ with the action $X(v+w) = Xv+Xw$ is a $\g$-mod
		\item $V^*$ with the action $(Xf)(v) = -f(Xv)$ is a $\g$-mod
		\item $V \otimes W$ with the action $X(v\otimes w) = Xv \otimes w + v \otimes Xw$
		\item If $W \subset V$, then $V/W$ with $X(\overline{v}) = \overline{Xv}$ is a $\g$-mod
		\item $\text{Hom}_\F(V,W)$ with the action $(Xf)(v) = X(f(v)) - f(Xv) $
	\end{enumerate}
	\label{module composition}
\end{prop}
\begin{proof}
	A. and B. are direct in all cases, the only problem is showing C.
	\begin{enumerate}
		\item 
		\begin{align*}
		[X,Y](v+w) &=[X,Y]v + [X,Y]w = X(Yv) - Y(Xv) + X(Yw)-X(Yw)\\ &= X(Yv+Yw) - Y(Xv+Xw) = X(Y(v+w)) - Y(X(v+w))
		\end{align*}
		\item Note that the minus sign is necessary.
		\begin{align*}
		([X,Y]f)v &= -f([X,Y]v) = -f(X(Yv)-Y(Xv)) = f(Y(Xv))-f(X(Yv)) \\&= -(Yf)(Xv) + (Xf)(Yv) = (X(Yf))(v) - (Y(Xf))(v) = (X(Yf)-Y(Xf))(v)
		\end{align*}    
		\item 
		\begin{align*}
		[X,Y](v\otimes w) &= ([X,Y]v)\otimes w + v \otimes ([X,Y]w)\\ &= X(Yv) \otimes w - Y(Xv) \otimes w + v \otimes X(Yw) - v \otimes (Y(Xw))\\
		&= X(Yv) \otimes w + v \otimes X(Yw) + Xv \otimes Yw + Yv \otimes Xw  \\
		& - Y(Xv) \otimes w - v \otimes Y(Xw) - Xv \otimes Yw - Yv \otimes Xw\\
		&= X(Yv \otimes w) + X(v \otimes Yw) - Y(Xv \otimes w) - Y(v \otimes Xw)\\ &= X(Y(v\otimes w)) - Y(X(v\otimes w))
		\end{align*}
		\item The only problem proving this one is that the action is well defined, in fact if $\overline{v} = \overline{w}$ then $\overline{Xv} - \overline{Xw} = \overline{X(v-w)} = \overline{0}$ since $\g W \subset W$

		\item The action defined is justified in the case of finite dimension by the trivial isomorphism between $\text{Hom}_\F(V,W)$ and $V^*\otimes W$
		\begin{align*}
		(X(Yf))(v) &= X((Yf)(v)) - (Yf)(Xv) = X(Yf(v)-f(Yv)) - Y(f(Xv)) - f(YXv)\\
		(Y(Xf))(v) &= Y(Xf(v)-f(Xv)) - X(f(Yv)) - f(XYv)\\
		[X(Yf)-Y(Xf)](v)&= X(Yf(v)) - Y(Xf(v)) - [f(X(Yv))-f(Y(Xv))] =\\
		&= [X,Y]f(v) - f([X,Y]v) = ([X,Y]f)(v)
		\end{align*}
	\end{enumerate}
\end{proof}\\
Now we shift our focus to the structure of modules themselves as algebraic structures, vector spaces embedded with a $\g$-action that is compatible with the bracket, the basic algebraic structure that is necessary to prove Weyl's Irreducibility Theorem can be summarized as follows:
\begin{defi}
A sub-module of a module $V$ is a subspace $W$ of $V$ such that $\g W \subset W$.\\
A linear transformation $T$ between modules $V$ and $W$ is said to be a module morphism if $T(Xv)=XT(v)$ for any $X \in \g$.
\label{module algebra}
\end{defi}

\begin{prop}
	Let $T:V\rightarrow W$ be a module morphism, then $\ker T$ is a sub-module of $V$
	\label{morphism}
\end{prop}
\begin{proof}
	$\ker T$ is a subspace of $V$, now for any $X \in \g$ and $v \in \ker T$ 
	$$T(Xv) = XT(v) = X0= 0$$
	Therefore $Xv \in \ker T$ and the proof is done.
\end{proof}
\begin{defi}
	Regarding the sub-structure of a particular module:\\
	A module $V$ is said to be \textbf{irreducible} if it has no proper and non-trivial sub-modules.\\
	A module $V$ is said to be \textbf{indecomposable} if it cannot be decomposed as a direct sum of any two of its proper sub-modules, that is if $V=M\oplus N$ for $M$ and $N$ sub-modules then $M=0$ or $N=0$.\\
	A module $V$ is said to be \textbf{reducible} if it is not irreducible.
\label{module-types}
\end{defi}
\subsection*{Initial Results on Representations}
\begin{lema}
	Let $\rho:\g\rightarrow \gl(V)$ be a representation of a semi-simple Lie Algebra $\g$, then $\rho(\g)\subset \gsl(V)$.
	\label{trace0lemma}
\end{lema}
\begin{proof}
	Since $\g$ is semi-simple then $\g=[\g,\g]$, since $[\g,\g] = \bigoplus [\g,\g_i] = \g$ by simplicity.\\ 
	Now since $\g=[\g,\g]$ then we can represent an element of $\g$ by $[X,Y]$ and therefore $\Tr(\rho[X,Y]) = \Tr(\rho(X)\rho(Y)-\rho(Y)\rho(X))=0$.\\
	In particular, if $V$ is one-dimensional then $\rho(\g) = 0$
\end{proof}
\begin{lema}[Schur's Lemma]
	Let $V$ be a irreducible $\g$-module over an algebraically closed field $\F$ and $\rho:\g \rightarrow \gl(V)$ be the correspondent representation, then the only endomorphism that commutes with $\rho(\g)$ are the scalars
	\label{Schur's Lemma}
\end{lema}
\begin{proof}
	If $T$ be a non-trivial endomorphism that commutes with $\rho(\g)$, then $T$ is a module morphism, and therefore its kernel must be $0$ since it is a sub-module of $V$.\\
	We also know that $T - \lambda I$ commutes with $\rho(\g)$ for all $\lambda \in \F$, therefore $T-\lambda I$ is either $0$ or an isomorphism, but since the field is algebraically closed, $T$ has an eigenvalue, and in that case the kernel can't be $0$, therefore $T-\lambda I = 0 \Rightarrow T=\lambda I$ for some $\lambda$.
\end{proof}\\
Now we will proceed to introduce the idea of a special element with respect to a representation, classically called the Casimir element.\\
Let $\beta:\g \times \g \rightarrow \F$ be any associative non-degenerate form, then for every basis $A=\{X_1,\cdots,X_n\}$ of $\g$ there exists a basis $B=\{Y_1,\cdots,Y_n\}$ such that $\beta(X_i,Y_j)=\delta_{ij}$. In which case writing $[X,X_i] = \sum_j a_{ij} X_j$ and $[X,Y_i]=\sum_j b_{ij} Y_j$. Note that
$$a_{ik} = \sum_{j=1}^n a_{ij}\delta_{jk} = \sum_{j=1}^n a_{ij} \beta(X_j,Y_k) =\beta([X,X_i],Y_k) = -\beta (X_i,[X,Y_k]) = -b_{ki} $$
Now if $\rho:\g \rightarrow \gl(V)$ is a one-to-one representation of a semi-simple $\g$, then the trace-form $\beta(X,Y)=\Tr(\rho(X)\rho(Y))$ satisfies the condition since $\rho(\g) \sim \g$ and $\text{rad} \beta$ is an ideal of $\g$ therefore $\beta$ is non-degenerate.\\
Now we define $c_\rho \in \gl(V)$ as $c_\rho = \displaystyle\sum_{i=1}^n \rho(X_i)\rho(Y_i)$ in the case of a one-to-one representation, then $\Tr(c_\rho) = \displaystyle \sum_{i=1}^n\Tr(\rho(X_i)\rho(Y_i))= n = \dim \g$
One of the important properties of $c_\rho$ is that it commutes with $\rho(\g)$, in fact, using $X_i=\rho(X_i)$ for simplicity in the proof:
\begin{align*}
Xc_\rho - c_\rho X &= \sum_{i=1}^n XX_iY_i - X_iY_iX = \sum_{i=1}^nXX_iY_i - X_iXY_i + X_iXY_i - X_iY_iX  \\
&=  \sum_{i=1}^n [X,X_i]Y_i + X_i[X,Y_i] = \sum_{i,j=1}^n a_{ij}X_jY_i + \sum_{i,j=1}^n b_{ij}X_iY_j = 0
\end{align*}
In case of a representation that is not one-to-one, we can consider another Lie Algebra $\g'=\g/\ker \rho$ and a new representation $\rho': \g' \rightarrow \gl(V)$ that satisfies the given conditions, since $\rho(\g)=\rho(\g')$ then $c_\rho'$ still commutes with $\rho(\g)$.
With this discussion we can summarize the properties of the Casimir element of a representation:
\begin{prop}
Given finite-dimensional semi-simple Lie Algebra $\g$, and a representation $\rho:\g\rightarrow \gl(V)$, then there exists an element $c_\rho \in \gl(V)$ that commutes with $\rho(\g)$ and has non-zero trace.Moreover, if $\rho$ is irreducible, then $c_\rho$ acts as a scalar(Schur's Lemma).
\end{prop}
\subsection*{Weyl's Theorem of Irreducibility}
\begin{teo}
	Let $V$ be a finite dimensional $\g$-mod, where $\g$ is a finite dimensional semi-simple Lie Algebra, then every proper sub-module $W\subset V$ admits a sub-module complement $W'$ such that $V=W\oplus W'$
	\label{Weyl's Theorem}
\end{teo}
\begin{proof}
	This proof proceeds by various cases and the use of induction.\\
	If $W$ is an irreducible co-dimension one sub-module of $V$, then $V/W$ is a $\g$-mod of dimension one, since $\g$ is semi-simple then $\g V \subset W$ and then since $c_\rho$ is a $\g$-mod morphism then $\ker c_\rho$ is a module, but $\ker c_\rho$ can't be $0$ because $\Tr c_\rho = \dim V \not=0$, and since $c:V\rightarrow W$($c$ being in the span of $\rho(\g)^2$), and $c$ acting as a scalar in $W$ by Schur's Lemma then $\ker c_\rho$ is the desired complement to $W$.\\
	If $W$ is a reducible co-dimension one sub-module of $V$, then let $W'\subset W$ be a sub-module, then $V/W$ is a sub-module and in fact it's immersed in the sub-module $V/W'$ with co-dimension one, since $\dim(V/W')<\dim(V)$ then the existence of a complement to $V/W$ in $V/W'$ proceeds by induction.\\
	Now let $W$ be any sub-module, consider $H=\text{Hom}(V,W)$ viewed as a $\g$-module, and let $\mathcal{V}$ be the subspace of $H$ consisting of maps whose restriction to $W$ is a scalar, that is $\mathcal{V}=\{f \in H | f(w)=a\cdot w \text{ for all } w \in W\}$ and let $\mathcal{W}$ be the ones whose restriction to $W$ are $0$. These are both sub-modules of $H$ and clearly $\mathcal{W} \subset \mathcal{V}$ is a subspace of co-dimension one(since their complement is determined by a scalar), to prove the fact that they are sub-modules, let $f \in \mathcal{V}$, $w \in W$ and $X \in \g$, then:
	$$(Xf)(w) = X(f(w)) - f(Xw) = X(a\cdot w)-a\cdot (Xv) = 0.$$
	And therefore $Xf \in \mathcal{W}$, therefore $\mathcal{W}$ has a complement in $\mathcal{V}$, let this complement be spanned by a particular $f$, by Lemma 1.6.5, $\g$ acts on $f$ trivially and therefore $(Xf)(v) = 0 \Rightarrow X(f(v)) - f(Xv)=0$, i.e that $f$ is a $\g$-mod morphism. Finally, since $f$ sends $V$ into $W$ and is a scalar in $W$, then $V=W \oplus \ker f$. 
\end{proof} 
\chapter{Root Decomposition}
This chapter presents the main results on the study of semi-simple Lie Algebras from an algebraic perspective, the same approach can be found in Humphreys \cite{humphreys1}.
\section{Abstract Jordan Decomposition}
There is a natural way to extend the Jordan decomposition of finite-dimensional operators to a given semi-simple Lie Algebra by the adjoint representation, the main idea is proving the existence of elements in a Lie Algebra that satisfy the conditions of the decomposition.\\
\begin{lema}
Every derivation in a semi-simple Lie Algebra is inner, meaning that if $D$ is a derivation then there exists $Y \in \g$ such that $D=\ad(Y)$
\label{InnerDerivations}
\end{lema}
\begin{proof}
Given $D$ a derivation, then the linear functional in $\g^*$ given by $f(X) = \Tr(D\ad(X))$ has a representational element $Y \in \g$ in such a way that $\kappa(X,Y) = f(X)$. And therefore $D=\ad(Y)$ because the element $\tilde{D} = D - \ad(Y)$, this relation implies $\Tr(\tilde{D}\ad(X))=0$ for any $X \in \g$. Now taking $X,Z \in \g$ arbitrary, then since:
$$[D,\ad(X)](Z) = D(\ad(X)(Z)) - \ad(X)DZ = D[X,Z] - [X,DZ] = [DX,Z] = \ad(DX)(Z)$$
Then:
\begin{align*}
\kappa(EX,Z) &= \Tr(\ad(EX)\ad(Z)) = \Tr([E,\ad(X)]\ad(Z))\\
&= \Tr(E\ad(X)\ad(Z) - \ad(X)E\ad(Z)) = \Tr(E\ad[X,Z])\\
&= 0
\end{align*}
Therefore since $Z$ is arbitrary $EX=0$ for any $X$ which implies that $E=0$, finally $D=\ad(Y)$
\end{proof}
\begin{lema} 
	If $D$ is a derivation and $\lambda,\mu\in \F$ then for every $X,Y \in \g$:
	$$(D-(\lambda+\mu)I)^n[X,Y] = \sum_{i=0}^n {n \choose i}[(D-\lambda I)^{n-i}X,(D-\mu I)^iY]$$
	\label{LeibnizProductFormula}
\end{lema}
\begin{proof}
	Proceeding by induction, the basis being the case $n=1$: 
	\begin{align*}
	(D-(\lambda+\mu)I)[A,B] &= D[A,B] - (\lambda + \mu)[A,B] \\ 
	&= [DA,B] + [A,DB] - [\lambda A,B] - [A,\mu B]\\
	&= [(D-\lambda I)A,B] + [A,(D-\mu I)B] 
	\end{align*}
	\begin{align*}
	(D-(\lambda + \mu)I)
	(D-(\lambda+\mu)I)^n[X,Y] &= (D-(\lambda+\mu)I)\sum_{i=0}^n {n \choose i}[(D-\lambda I)^{n-i}X,(D-\mu I)^iY]\\
	&=\sum_{i=0}^n {n\choose i}[(D-\lambda I)^{n-i+1}X,(D-\mu I)^i Y]\\
	&+\sum_{i=0}^n {n \choose i}[(D-\lambda I)^{n-i}X,(D-\mu I)^{i+1}Y]\\
	&=\sum_{i=1}^{n} \left({n \choose {i}}[(D-\lambda I)^{n+1-i},(D-\mu I )^i Y]\right) + [(D-\lambda I)^{n+1}X,Y]\\
	&+\sum_{i=1}^{n}\left({n \choose i-1}[(D- \lambda I)^{n+1-i},(D-\mu I)^i Y] \right) + [X,(D- \mu I)^{n+1}Y]\\
	&= \sum_{i=0}^{n+1}{{n+1}\choose i}[(D-\lambda I)^{n+1-i}X,(D-\mu I)^i Y]
	\end{align*}
\end{proof}
\begin{corol}
	On an algebraically closed field, if $D=S+N$ is the Jordan decomposition of a derivation, then $S$ and $N$ are derivations
\end{corol}
\begin{proof}
	Let $\g = \displaystyle\bigoplus_{\alpha \in \F} \g_\alpha$ be the generalized eigenspace space decomposition of $\g$, then the formula above shows that $[\g_\alpha,\g_\beta] \subset \g_{\alpha+\beta}$, and therefore if $X \in \g_\alpha$ and $Y \in \g_\beta$ then:
	$$S[X,Y] = (\alpha+\beta)[X,Y] = \alpha[X,Y]+\beta[X,Y] = [SX,Y]+[X,SY]$$
	And therefore $S$ is a derivation since $\g$ is the sum of eigenspaces, it follows that $N=D-S$ is a derivation.
\end{proof}
\begin{prop}
	For every $X \in \g$ there exists unique $S,N \in \g$ satisfying the following conditions:
	\begin{enumerate}[label=\Alph*.]
		\item $X=S+N$
		\item $\ad\  S$ is diagonizable and $\ad\  N$ is nilpotent
		\item $[S,N]=0$
	\end{enumerate}
	\label{AbsJordanDecomp}
\end{prop}
\begin{proof}
Since $\ad(X)$ is a derivation, then its semi-simple part and nilpotent part are derivations and therefore are adjoints of elements in $\g$, let those elements be $S$ and $N$, $\ad(X)=\ad(S)+\ad(N)$.\\
Since $\g$ is semi-simple, then the adjoint representation is one-to-one(its kernel is $\z(\g)=0$) and therefore $\ad(X)=\ad(S+N) \Rightarrow X=S+N$\\
Finally, since $[\ad(S),\ad(N)]=0$ then $\ad[S,N]=0$ and therefore $[S,N]=0$.
\end{proof}
\begin{prop}
	If $\g$ is a semi-simple Lie Algebra and $\varphi:\g \rightarrow \gl(V)$ is a representation, then for any element $X \in \g$ with Abstract Jordan Decomposition $X=S+N$ then the Jordan Decomposition of $\varphi(X)$ is $\varphi(X)=\varphi(S)+\varphi(N)$.\\
	In particular the representation of any semi-simple element is semi-simple and of every nilpotent element is nilpotent.
	\label{PreservationOfAbsJordan} 
\end{prop}
\section{Toral Sub-algebras}
Considering the Jordan Decomposition of a Lie Algebra, we want a natural way to consider subalgebras of semi-simple element, in fact, if $\g$ is a finite dimensional Lie Algebra over a field $\F$, then since $\g$ is not nilpotent, there exists at least one element that is semi-simple. We want to analyze the structure of the following object:
\begin{defi}[Toral Subalgebras]
A subalgebra of a semi-simple Lie Algebra that consists only of semi-simple elements(with respect to the abstract Jordan Decomposition) is called a Toral subalgebra.
\label{toralsubalg Def}
\end{defi}
The existence of such a subalgebra has already been discussed, it has some nice properties that allow us to decompose a semi-simple Lie Algebra based on a maximal toral subalgebra.
\begin{prop}
	If $\h\le\g$ is a toral sub-algebra, then $\h$ is abelian.
	\label{abeliantoral}
\end{prop}
\begin{proof}
	We shall prove that $\ad_\h(X)=0$ for any $X \in \h$. If this is not the case, then since $X$ is semi-simple there exists an eigenvector of $\ad_\h(X)$ with non-zero eigenvalue $\alpha$, let this be $H$, then $\ad(X)(H) = \alpha H \Rightarrow \ad(H)(X) = -\alpha H\Rightarrow \ad(H)^2 (X)=0$. Meaning that $\ad(H)(X)$ is an eigenvector of $\ad(H)$ with eigenvalue $0$\\
	On the other hand, $H$ is semi-simple, therefore there is a basis ${Y_i} \subset \g$ of eigenvectors then $X = \sum \beta_i Y_i$, applying $\ad(H)$ to this relation we see that $\ad(H)X$ is a sum of non-zero eigenvectors or $0$, contradicting the fact that $\ad(H)(X)$ is an eigenvector with eigenvalue $0$ or the fact that $H\not=0$.
\end{proof}\\
\begin{prop}
If $\h$ is a maximal toral subalgebra, then it's possible to decompose $\g$ with respect to $\h^*$, for $\alpha \in \h^*$ define $$\g_\alpha = \{X \in \g | [H,X]=\alpha(H)X \text{ for all } H \in \h\}$$
Then these spaces satisfy $[\g_\alpha,\g_\beta]\subset \g_{\alpha+\beta}$ for any $\alpha,\beta \in \h^*$ and moreover, if $\alpha+\beta \not=0$ then $\kappa(\g_\alpha,\g_\beta)=0$.
\label{OrthogonalityAndSum}
\end{prop}
\begin{proof}
	Since all elements of $\ad H$ are commuting semi-simple endomorphisms, then they are all diagonal with respect to a basis of $\g$, in this case we can do the decomposition 
	$$\g = \bigoplus_{\alpha \in \h^*} \g_\alpha$$
	Now fixing $X \in \g_\alpha$ and $Y \in \g_\beta$, then:
	$$[H,[X,Y]] = [[H,X],Y]+[X,[H,Y]] = \alpha(H) [X,Y] + \beta(H) [X,Y] = (\alpha+\beta)(H)[X,Y]$$
	Which implies that $[X,Y] \in \g_{\alpha+\beta}$.\\
	For the remaining assertion, consider that for $X \in \g_\alpha$, $Y \in \g_\beta$ and $H \in \h$ then:
	\begin{align*}
	\kappa([H,X],Y)&=\alpha(H)\kappa(X,Y)\\
	\kappa(X,[H,Y])&=\beta(H)\kappa(X,Y)\\
	\kappa([H,X],Y) = -\kappa(X,[H,Y]) &\Rightarrow (\alpha + \beta)(H)\kappa(X,Y) = 0
	\end{align*}
\end{proof}
\begin{corol}
	The restriction of $\kappa$ to $\g_0$ is non-degenerate
	\label{g0killingnondeg}
\end{corol}
\begin{proof}
	If it was, then let $X \in \g_0$ be such that $\kappa(X,\g_0)=0$, but in that case by the previous relation $\kappa(X,\g_\alpha) =0$ for any $\alpha\not=0$, since $\g= \bigoplus \g_\alpha$ then, $\kappa(X,\g)=0$, since the killing form of $\g$ is non-degenerate, we reached a contradiction.
\end{proof}
\begin{teo}
	Let $\h$ be a maximal toral subalgebra, then $\h=\g_0$
	\label{hg0}
\end{teo}
\begin{proof}
	We will proceed in steps:
	\begin{enumerate}
		\item $\g_0$ contains the nilpotent and semi-simple parts of its elements.\\
		To say that $X \in \g_0$ is to say that $\ad(X)(\h)=0$, but by Jordan decomposition properties, $\ad \ S$ and $\ad N$ must map $\h$ to $0$ (they are polynomials in $\ad(X)$)
		\item All semi-simple elements of $\g_0$ lie in $\h$.\\
		If $S \in \g_0$ is semi-simple and $[\h,S]=0$, then $\h+\F S$ is a toral subalgebra, therefore $S \in \h$ by maximality of $\h$.
		\item The restriction of $\kappa$ to $\h$ is non-degenerate.\\
		Let $H \in \h$ be such that $\kappa(H,\h)=0$, if $N \in \g_0$ is such that $N$ is nilpotent, then $[X,H]=0$ and $\ad(X)$ is nilpotent, therefore $\ad(N)\ad(H)$ is nilpotent and therefore $\Tr(\ad(N)\ad(H))=0 \Rightarrow \kappa(N,H)=0$, but then for all $X=S+N \in \g_0$, we have that $\kappa(H,X)=0$ since $S \in \h$ by $(2)$. contradicting Corollary \ref{g0killingnondeg}, since $H \in \g_0$.
		\item $\g_0$ is a nilpotent algebra. \\
		If $S \in \g_0$ is semi-simple, then $S \in \h$ and therefore $[S,\g_0]=0$, implying that $\ad(S)$ is nilpotent in $\g_0$. Now if $X=S+N$ is any element of $\g_0$ then $\ad(X)$ is the sum of commuting nilpotent endomorphisms, and therefore $\ad(X)$ is nilpotent, by Engel's Theorem $\g_0$ is nilpotent
		\item $\h \cap [\g_0,\g_0]=0$.\\
		Since $[\h,\g_0]=0$ then $\kappa(\h,[\g_0,\g_0]) =0$ by associativity, therefore $[\g_0,\g_0]\not \in \h$ by $(3)$
		\item $\g_0$ is abelian.\\
		Otherwise $[\g_0,\g_0]\not=0$, since $\g_0$ is nilpotent then $[\g_0,\g_0]\cap\z(\g_0)\not=0$, let $X$ be an element in this interception, then its nilpotent part $N$ is non-zero and also lies in $Z(\g_0)$ (since $\ad\ N$ is a polynomial in $\ad \ X$), but then since $\ad \ N$ is nilpotent and commutes with $\g_0$ then $\kappa(N,\g_0)=0$, contradicting Corollary \ref{g0killingnondeg}
		\item $\h=\g_0$\\
		Otherwise, there exists a non-zero nilpotent element $N \in \g_0$, but in that case since $\g_0$ is abelian then for every $X \in \g_0$, $\ad \ N \ad \ X = \ad \ X \ad \ N$ is a nilpotent endomorphism and therefore has trace $0$, implying that $\kappa(N,\g_0)=0$ contradicting Corollary \ref{g0killingnondeg}
	\end{enumerate}
\end{proof}
\section{Finite Dimensional Representations of $\gsl(2)$}
Considering the $\gsl(2,\F)$ Lie Algebra over an algebraically closed field with characteristic $0$, a traditional basis considered of this algebra as a matrix algebra is the following:
$$X = \begin{pmatrix}
0 & 1 \\
0 & 0 
\end{pmatrix}, \ \ H = \begin{pmatrix}
1 & 0 \\
0 & -1 
\end{pmatrix}, \ \ Y = \begin{pmatrix}
0 & 0 \\
1 & 0 
\end{pmatrix}$$
The commutators in this algebra satisfy the following relations:
$$[X,Y]=H, \ \ [H,X]=2X, \ \ [H,Y]=-2Y$$
Considering the fact that $H$ is diagonal and that the preservation of the Jordan Decomposition implies that $H$ acts diagonally in any $\g$-mod $V$( Proposition \ref{PreservationOfAbsJordan}). Because of this, one can decompose $V$ as a sum of eigenspaces with respect to $H$, letting $V_\lambda = \{v \in V|Hv=\lambda v\}$ then if we assume $V$ to be finite-dimensional $V=\displaystyle\bigoplus_{\lambda \in \F} V_\lambda$. Whenever $V_\lambda \not=0$ we call $\lambda$ a weight and we call $V_\lambda$ a weight space.\\
\begin{lema}
	If $v \in V_\lambda$, then $Xv \in V_{\lambda+2}$ and $Yv \in V_{\lambda-2}$
\end{lema}
\begin{proof}
	Since $V$ is a $\g$-mod then $[A,B]v = A(Bv)-B(Av)$ for any $v \in V$ and $A,B \in \g$, therefore:
	$$H(Xv)=[H,X]v + X(Hv) = 2Xv + X(\lambda v) = (\lambda + 2)Xv$$
	$$H(Yv)=[H,Y]v + Y(Hv) = -2Yv + Y(\lambda v) = (\lambda - 2)Xv$$
\end{proof}\\
Since $V$ is finite-dimensional, we know that there exists some $\lambda$ in such a way that $V_\lambda\not=0$ such that $V_{\lambda+2}=0$, we will call one of these weights maximal and any vector in $V_\lambda$ as a maximal vector. One direct consequence of this definition is:
\begin{prop}
If $v$ is maximal vector, then $Xv=0$
\end{prop}
We can determine the action of $\g$ in a special subset of vectors, inspired by the idea of determining the action of $X$ by the action of $Y$:
\begin{lema}
Let $v_0$ be a maximal vector, and set $v_{-1}=0$ and $v_{i}=\frac{1}{i!}Y^iv_0$ for $i\ge 0$ then:
\begin{enumerate}[label=(\alph*)]
	\item $Hv_i = (\lambda - 2i)v_i$
	\item $Yv_i = (i+1)v_{i+1}$
	\item $Xv_i = (\lambda-i+1)v_{i-1}$ for $i\ge0$
\end{enumerate}
\end{lema}
\begin{proof}
	\begin{enumerate}[label=(\alph*)]
		\item Follows directly from Lemma $2.3.1$
		\item Is a consequence of the definition, in fact:
		$$v_{i+1} = \frac{1}{(i+1)!}y^{i+1}v_0 = \frac{1}{i+1} Y\left(\frac{1}{i!}Y^i v_0\right) = \frac{1}{i+1} Yv_i.$$
		\item Since $v_0$ is maximal then $Xv_0=0$ and the result is valid, proceeding by induction:
		\begin{align*}
		Xv_i &= \frac{1}{i}X(Yv_{i-1})\\
		iXv_i&= [X,Y]v_{i-1} + Y(Xv_{i-1})\\
			 &= Hv_{i-1} + Y(Xv_{i-1})\\
			 &= (\lambda-2(i-1))v_{i-1} + Y(\lambda - i + 2)v_{i-2}\\
			 &= (\lambda-2(i-1))v_{i-1} + (\lambda - i + 2)(i-1)v_{i-1}\\
			 &=  (\lambda - 2i + 2 + i\lambda - i^2 + 2i + i - 2)v_{i-1}\\
			 &= i(\lambda-i+1)v_{i-1}.
		\end{align*}
	\end{enumerate}
\end{proof}\\
Now if we consider $V$ to be irreducible, we can explicitly classify $V$ with respect to the $v_i$ which in turn are completely determined by $\lambda$.
\begin{teo}
If $V$ is irreducible, then the set of $\{v_i\}$ form a basis of $V$, $\lambda$ is a positive integer and the number of vectors $\{v_i\}$ is precisely $\lambda+1$
\end{teo}
\begin{proof}
	Since each $v_i$ is an eigenvector of $H$ with different eigenvalues$(a)$, they are linearly independent.\\
	The span of the set $\{v_i\}$ is closed under the action of $\g$, since this span is non-zero then it must be the whole $V$ since it is an irreducible module.\\
	Now let $m$ be the largest value such that $v_m \not=0$ but $v_{m+1}=0$ (possible since $V$ is finite dimensional), then letting $i=m+1$ in $(c)$ we find:
	$$Xv_{m+1} = (\lambda - (m+1)+1)v_m \iff 0 = (\lambda-m)v_m$$
	And therefore $\lambda=m$, which is a positive integer, furthermore $\{v_0,v_1,\cdots,v_m\}$ is a basis of $V$, and therefore $\dim V = \lambda + 1$
\end{proof}
We can summarize a classification of every  $\gsl(2)$ module based on this theorem:
\begin{teo}[Classification of $\gsl(2)$ modules]
If $V$ is an irreducible module of $\gsl(2)$ of dimension $m+1$, then:
\begin{enumerate}[label=(\alph*)]
	\item $V=V_{-m} \oplus V_{-m+2} \oplus \cdots \oplus V_{m-2} \oplus V_m$, each with dimension one.
	\item $V$ has a unique maximal weight and a unique maximal vector up to scalar multiples.
	\item The $\gsl(2)$-action is defined by the formulas in Lemma $2.3.3$, in particular, there exists at most one irreducible module of $\gsl(2)$ of dimension $m$ up to isomorphism.
\end{enumerate}
\end{teo}
Since every $\gsl(2)$ module is the sum of irreducible modules (Weyl's Theorem \ref{Weyl's Theorem}), then:
\begin{corol}
	Let $V$ be any finite-dimensional $\gsl(2,\F)$ module, then all the eigenvalues of $H$ on $V$ are integers and each occurs along with its negative an equal number of times. Moreover, in any decomposition of $V$ as irreducible sub-modules, the number of modules is $\dim(V_0)+\dim(V_1)$
	\label{sl2modules}
\end{corol}
\begin{proof}
	The first result is direct. Since every irreducible module is a sum of weight spaces with distance $2$ from each other, then each irreducible sub-module must have weight $0$ or $1$, but not both.
\end{proof}
\section{Root Decomposition}
\subsection*{Orthogonality}
Let $\g$ be a finite-dimensional semi-simple Lie Algebra over $\F$ algebraically closed with characteristic $0$.\\
Fixing $\h$ a maximal toral subalgebra of $\g$, if $\g_\alpha = \{X \in \g | [H,X]=\alpha(H)x\}$ \text{ for all } $H \in \h$ \text{ and } $\alpha \in \h^*\}$ we'll call the subset of $\h^*$ such that $\g_\alpha \not=0$, except $0$ as $\Phi$ and its elements roots. \\
Since the killing form restricted to $\h$ is non-degenerate, then for every $\alpha \in \h^*$ we can find some $T_\alpha \in \h$ that represents it, meaning $\alpha(H)=\kappa(T_\alpha,H)$.\\
Reminding ourselves of the property of orthogonality between roots, meaning that if $\alpha+\beta \not=0$ for elements in $\h^*$ then $\kappa(\g_\alpha,\g_\beta)=0$.\\
With this we can summarize the first properties of root decomposition:
\begin{prop}
	\begin{enumerate}[label=(\alph*)]
		\item $\Phi$ spans $\h^*$
		\item If $\alpha \in \Phi$ then $-\alpha \in \Phi$
		\item Let $\alpha \in \Phi$, $X \in \g_\alpha$ and $Y \in \g_{-\alpha}$ then: $[X,Y]=\kappa(X,Y)T_\alpha$
		\item If $\alpha \in \Phi$, then $[\g_\alpha,\g_{-\alpha}]$ is one dimensional
		\item $\alpha(T_\alpha)=\kappa(T_\alpha,T_\alpha)\not=0$
		\item If $\alpha \in \Phi$ and $X_\alpha \in \g_\alpha$ is any non-zero element, then there exists $Y_\alpha \in \g_\alpha$ such that $\{X_\alpha,[X_\alpha,Y_\alpha],Y_\alpha\}$ spans a three dimensional subalgebra of $\g$ isomorphic to $\gsl(2)$, call $[X_\alpha,Y_\alpha]=H_\alpha$
		\item $H_\alpha = -H_{-\alpha} = \frac{2T_\alpha}{\kappa(T_\alpha,T_\alpha)}$ 
	\end{enumerate}
	\label{Orthogonality}
\end{prop}
\begin{proof}
	\begin{enumerate}[label=(\alph*)]
		\item If $\Phi$ fails to span $\h^*$, then by duality we get a non-zero element $H \in \h$ such that $\alpha(H)=0$ for all $\alpha \in \Phi$, but in that case for any $X_\alpha \in \g_\alpha$ this means $[H,X_\alpha]=\alpha(H)X_\alpha =0$.
		But $[H,\h]=0$, and this implies that $H \in \z(\g)$, a contradiction
		\item Otherwise, meaning that $\alpha \in \Phi$ but $-\alpha \not \in \Phi$, then there is not an element $\beta \in \Phi$ such that $\alpha + \beta = 0$, meaning that $\kappa(\g_\alpha,\g)=0$, contradicting the non-degeneracy of $\kappa$
		\item Let $H \in \h$ be arbitrary, then:
		\begin{align*}
		\kappa(H,[X,Y]) &= \kappa([H,X],Y) = \alpha(H)\kappa(X,Y)=\kappa(H,T_\alpha)\kappa(X,Y)=\kappa(H,\kappa(X,Y)T_\alpha)\\
		\kappa(H,[X,Y]-\kappa(X,Y)T_\alpha)&=0
		\end{align*}
		This in turn implies that $\h$ is orthogonal to $[X,Y]-\kappa(X,Y)T_\alpha$, forcing $[X,Y]=\kappa(X,Y)T_\alpha$ since $[X,Y] \in \g_0 = \h$ and $\kappa$ is non-degenerate in $\h$.
		\item $(c)$ shows that $T_\alpha$ spans $[\g_\alpha,\g_{-\alpha}]$, provided it is not $0$, and this space can't be $0$ because otherwise $\kappa(\g_\alpha,\g_{-\alpha})=0$ and therefore $\kappa(\g_\alpha,\g_{-\alpha}=0)$
		\item If $\alpha(T_\alpha)=0$ choose $X \in \g_\alpha$ and $Y_\alpha$ such that $\kappa(X,Y)=1$ (possible due to $(d)$), then the subspace $S$ spanned by ${X,T_\alpha,Y}$ satisfies $[X,Y]=T_\alpha$, $[X,T_\alpha]=0$ and $[Y,T_\alpha]=0$, therefore its a three dimensional solvable Lie Algebra, then $S \sim \ad S \subset \gl(\g)$. Then $\ad[S,S]$ is nilpotent and therefore $\ad T_\alpha$ is both semi-simple and nilpotent, implying that $\ad(T_\alpha)=0 \Rightarrow T_\alpha=0$, absurd. 
		\item Let $H_\alpha = \frac{2T_\alpha}{\kappa(T_\alpha,T_\alpha)}$, then find $Y_\alpha \in \g_{-\alpha}$ such that $\kappa(X,Y) = \frac{2}{\kappa(T_\alpha,T_\alpha)}$ then $[X_\alpha,Y_\alpha] = \kappa(X,Y)T_\alpha = H_\alpha$.
		Now $[H_\alpha,X_\alpha] = \alpha(H_\alpha)X_\alpha = \frac{2\alpha(T_\alpha)}{\kappa(T_\alpha,T_\alpha)}X_\alpha = 2X_\alpha $, and similarly for $[H,Y_\alpha]=-2Y_\alpha$.
		\item Remember that $T_{-\alpha}$ is defined as the element that satisfies $\kappa(T_{-\alpha},H)=-\alpha(H)$ for any $H \i \h$, but $\kappa(-T_\alpha,H)=-\kappa(T_\alpha,H)=-\alpha(H)$, therefore $T_{\alpha}=-T_{-\alpha}$ and it follows that $H_{\alpha}=-H_{-\alpha}$
		\end{enumerate}
	\end{proof}
\subsection*{Integrality}
For each pair of roots $\alpha,-\alpha$ let $S_\alpha$ be a subalgebra isomorphic to $\gsl(2)$ constructed above. A lot of properties of its representations have been established, using those properties we can analyze some modules contained in $\g$, given rise to the following:
\begin{prop}
	Let $\alpha \in \Phi$ and $M_\alpha=\displaystyle\bigoplus_{c \in\F} \g_{c\alpha}$, then $M_\alpha$ is a $S_\alpha$-module and:
	\begin{enumerate}[label=(\alph*)]
		\item $\dim\g_\alpha = 1$
		\item If $c\alpha$ is a root, then $c=\pm 1$.
	\end{enumerate}
	\label{IntegralityM}
\end{prop}
\begin{proof}
	Since $[\g_{c\alpha},\g_{c'\alpha}]\subset \g_{(c+c')\alpha}$ and $S_\alpha \subset M_\alpha$ then $M_\alpha$ is a $S_\alpha$-module via the adjoint.\\
	Now the weights of $H_\alpha$ in $M$ are $2c\alpha$ since $\alpha(H_\alpha)=2$ and $[H_\alpha,X] = c\alpha(H_\alpha)X$ for any $X \in \g_{c\alpha}$.\\
	Now consider that $\ker \alpha$ is a sub-module of $M$ with codimension 1 in $\h$ complementary to $H_\alpha$, therefore the weight $0$ only occurs in $\ker \alpha$ and $S_\alpha$, but $S_\alpha$ is irreducible and therefore the only even weights in $M$ are $0,\pm 2$. This proves that twice a root can never be a root, but then half a root can't be a root either, therefore there's no weight $1$ in $M$. This in turn implies that $M=\ker \alpha \oplus S_\alpha$, and in particular $\dim\g_\alpha=1$ and that the only multiples of $\alpha$ which are roots are $\pm \alpha$.
\end{proof}
\begin{prop}
	Let $\alpha,\beta \in \Phi$ such that $\beta \not= \pm \alpha$ and let:
	$$K_\alpha(\beta) = \g_{\beta - r \alpha} \oplus \g_{\beta - (r-1)\alpha} \oplus \cdots \oplus \g_{\beta} \oplus \g_{\beta+1} \oplus \cdots \g_{\beta + q_\alpha}$$
	Where $r$ is the biggest number such that $\beta - r\alpha$ is a root and $q$ is the biggest such that $\beta + q\alpha$ is a root. \\
	Then $K_\alpha$ is a $S_\alpha$-module and :
	\begin{enumerate}
		\item $\beta(H_\alpha) \in \mathbb{Z}$ and $\beta-\beta(H_\alpha)\alpha \in \Phi$
		\item If $i$ is such that $-r\le i \le q$ then $\beta + i \alpha \in \Phi$ and $\beta(H_\alpha) = r-q$
	\end{enumerate}
	\label{IntegralityK}
\end{prop}
\begin{proof}
	Each root space is one dimensional, and none of the $\beta+i\alpha$ can equal $0$ since $\beta \not= \pm \alpha$ and no other multiple of $\alpha$ is a root.\\
	Now if $X \in \g_{\beta+i\alpha}$ then $[H_\alpha,X]=(\beta+i\alpha)(H_\alpha)X = (\beta(H_\alpha)+2i)X$
	And therefore the only distinct weights of $H_\alpha$ are $\beta(H_\alpha)+2i$, since all of those must be integers(Prop. \ref{sl2modules}), then $\beta(H_\alpha) \in \mathbb{Z}$. Obviously not both $0$ or $1$ can occurs as weights, and since every root space is one dimensional then $K_\alpha$ is irreducible ($\dim({K_\alpha}_0)+\dim({K_\alpha}_1)=1)$.\\
	Since every weight occurs along with its negative, then:\\
	$$\beta(H_\alpha)-2r=-(\beta(H_\alpha)+2q) \Rightarrow \beta(H_\alpha) = r-q$$
\end{proof}
\subsection*{Rationality}
The killing form is non-degenerate and symmetric, furthermore its restriction to a maximal toral subalgebra is connected to integers in some way, we want to construct an inner product based in it.\\
Let $(\gamma,\delta) = \kappa(T_\gamma,T_\delta)$ for all $\gamma,\delta \in \h^*$ and $\{\alpha_1,\cdots,\alpha_\ell\}$ a base consisting of roots, and let for any $\beta \in \h^*$, it's written in this basis as
\begin{equation}
 \beta = \sum_{i=1}^\ell c_i\alpha_i
\end{equation}
\begin{prop}
If $\beta \in \delta$, then $c_i \in \mathbb{Q}$
\end{prop}
\begin{proof}
We know that $$\beta(H_\alpha) = \kappa(T_\beta,H_\alpha) = \frac{2\kappa(T_\beta,T_\alpha)}{\kappa(T_\alpha,T_\alpha)} = \frac{2(\beta,\alpha)}{(\alpha,\alpha^)} \in \mathbb{Z}$$
So using equation $(2.1)$ with this fact in mind we get:
\begin{align}
	\frac{2(\beta,\alpha_j)}{(\alpha_j,\alpha_j)} &= \sum_{i=1}^\ell c_i \frac{2(\alpha_j,\alpha_i)}{(\alpha_j,\alpha_j)}\\
	\beta(H_{\alpha_j}) &= \sum_{i=1}^\ell c_i\alpha_i(H_{\alpha_j})
\end{align}
Therefore since this is a linear equation in $c_i$ with integer coefficients solvable in $\F$, then it's solvable in $\mathbb{Q}$, implying that $c_i \in \mathbb{Q}$.
\end{proof}\\
Let $E_\mathbb{Q}$ be the space spanned in $\mathbb{Q}$ by the roots, we just proved that $\dim(E_\mathbb{Q}) = \ell$ 
Even more is true to this:
\begin{prop}
	The form $(,)$ is naturally extended to $E_\mathbb{Q}$ and is positive definite
\end{prop}
\begin{proof}
	Notice that:
	\begin{align*}
		(\lambda,\mu) &= \kappa(T_\lambda,T_\mu) = \Tr(\ad T_\lambda\ad T_\mu) = \sum_{\alpha \in \Phi} \alpha(T_\lambda)\alpha(T_\mu)\\
		&=\sum_{\alpha \in \Phi}(\alpha,\lambda)(\alpha,\mu)
	\end{align*}
	In particular $(\beta,\beta) = \sum (\alpha,\beta)^2$ 
	Multiplying this relation by $\frac{4}{(\beta,\beta)^2}$ we get
	$$ \frac{4}{(\beta,\beta)}=\sum_{\alpha \in \Phi}\left(\frac{2(\alpha,\beta)}{(\beta,\beta)}\right)^2 \in \mathbb{Z}$$
	Therefore $(\beta,\beta) \in \mathbb{Q}$ and in turn, since $\frac{2(\alpha,\beta) }{(\beta,\beta)} \in \mathbb{Z}$ then $(\alpha,\beta) \in \mathbb{Q}$, proving that the form is well defined in $E_{\mathbb{Q}}$ now since $(\beta,\beta)=\sum (\alpha,\beta)^2$ it is positive definite as the sum of squares of rationals.
\end{proof}
\subsection*{Summary}
Let $E$ be the real vector space extending the base field of $E_\mathbb{Q}$ from $\mathbb{Q}$ to $\mathbb{R}$, then the following properties are satisfied:
\begin{enumerate}[label=(\alph*)]
	\item $\Phi$ spans $E$ and $0$ does not belong to $\Phi$
	\item If $\alpha \in \Phi$ then the only other multiple of $\alpha$ in $\Phi$ is $-\alpha$.
	\item If $\alpha,\beta \in \Phi$ then $\displaystyle \beta - \frac{2(\beta,\alpha)}{(\alpha,\alpha)}\alpha \in \Phi$
	\item If $\alpha,\beta \in \Phi$, then $\frac{2(\beta,\alpha)}{(\alpha,\alpha)} \in \mathbb{Z}$
	\label{RootSystemLie}
\end{enumerate}

\chapter{Root Systems}
\section{Axiomatic}
We want to study Root Systems as appeared on Lie Algebras on their own, for such we are gonna define one as something that satisfies the conditions on the Summary portion of Section 2.4 .\\
\begin{defi}[Root System]
Given an Euclidean Space $E$ with inner product denoted by $\langle\ , \  \rangle$ and a finite subset $\Phi \subset E$, then the pair $(E,\Phi)$ is said to be a Root System if:
\begin{enumerate}
\item $\Phi$ spans $E$ and $0$ does not belong in $\Phi$
\item If $\alpha \in \Phi$ then the only other multiple of $\alpha$ in $\Phi$ is $-\alpha$
\item If $\alpha,\beta \in \Phi$ then $\beta-\frac{2\langle \beta,\alpha\rangle}{\langle \alpha,\alpha \rangle} \in \Phi$
\item If $\alpha,\beta \in \Phi$ then $\frac{2\langle \beta,\alpha\rangle}{\langle \alpha,\alpha \rangle} \in \mathbb{Z}$
\end{enumerate}
To reduce notation, we will define $\frac{2\alpha}{\langle \alpha,\alpha\rangle}$ as $\alpha^\lor$ and furthermore we will define $s_\alpha$ as the linear transformation 
$$s_\alpha(v) = v - \cartan{v}{\alpha}\alpha$$ therefore we can reduce 3. and 4. to:
\begin{enumerate}[label=\arabic*b.]
	\setcounter{enumi}{2}
	\item If $\alpha,\beta \in \Phi$ then $s_\alpha(\beta)\in \Phi$
	\item If $\alpha,\beta \in \Phi$ then $\cartan{\beta}{\alpha} \in \mathbb{Z}$
\end{enumerate}
\end{defi}
Some further notation, we call the operation $(\alpha,\beta)\mapsto \cartan{\beta}{\alpha}$ as the Cartan product 
It is important to note that axiom $3.$ has a nice geometric interpretation, that being: $\Phi$ is closed under reflections with respect to a hyperplane defined from a root.\\
\textbf{EXAMPLES TODO}\\
Axiom $4.$ allows us to deduce some simple integral restrictions to a root system:
\begin{prop}
	Let $\alpha,\beta \in \Phi$ with $\|\beta\|\ge \|\alpha\|$, denoting by $\theta$ the angle between then and the Cartan product is restricted to those in the following table:\\
	\begin{table}[ht]
		\caption{Possible Values of the Cartan product}
		\begin{center}
				\begin{tabular}{|c|c|c|}
					\hline
					$\cartan{\alpha}{\beta}$ & $\cartan{\beta}{\alpha}$ & 0 \\
					\hline
					0 & 0 & $90^\circ$\\
					$\pm 1$ & $\pm 1$  & $60^\circ$ or $120^\circ$ \\
					$\pm 1$ & $\pm 2$  & $45^\circ$ or $135^\circ$\\
					$\pm 1$ & $\pm 3$  & $30^\circ$ or $150^\circ$ \\
					$\pm 2$ & $\pm 2$  & $0^\circ$ or $180^\circ$\\
					\hline
				\end{tabular}
		\end{center}
		\label{Cartan}
	\end{table}
\end{prop}
\begin{proof}
	Consider that $\langle \alpha,\beta \rangle = \|\alpha\|\|\beta\|\cos\theta$ where $\theta$ is the angle between the roots, then since $\|\alpha^\lor\| = \frac{2}{\|\alpha\|}$ then:
	$$ \cartan{\alpha}{\beta}\cartan{\beta}{\alpha}= \frac{4\|\alpha\|\|\beta\|}{\|\beta\|\|\alpha\|}\cos^2\theta = 4 \cos^2 \theta$$ 
	Since $0\le \cos^2\theta \le 1$  then the only possibilities are with $\cartan{\alpha}{\beta}\cartan{\beta}{\alpha}\le 4$.\\
	Which are all the cases present in Table \ref{Cartan} excluding the cases $(0,n)$ with $n\not=0$ and $(1,4)$. \\
	The case $(0,n)\  n\not=0$ can be excluded since if $\cartan{\alpha}{\beta}=0$ then $\langle \alpha,\beta \rangle = 0$ which implies that $\cartan{\beta}{\alpha} = 0$\\
	The case $1,4$ can also be excluded because if $4\cos^2 \theta = 4$ then $\theta = 0^\circ$ or $\theta = 180^\circ$ which implies that $\beta$ is a multiple of $\alpha$, meaning that $\beta = \pm \alpha$ and therefore $\cartan{\alpha}{\beta} = \pm2$.\\
	Notice that the last case $(\pm 2,\pm 2)$ only occurs on proportional roots.
\end{proof}
An important corollary of this restriction is as follows:
\begin{corol}
	Let $\alpha,\beta$ be non-proportional roots, if $\langle \alpha,\beta \rangle > 0$ then $\alpha-\beta$ is a root and if $\langle \alpha,\beta \rangle <0$ then $\alpha+\beta$ is a root.
	\label{corolcartan}
\end{corol}
\begin{proof}
	Since $\langle \alpha, \beta \rangle$ is positive then Table \ref{Cartan} shows that either $\cartan{\alpha}{\beta}=1$ or $\cartan{\beta}{\alpha}=1$.\\
	$$\cartan{\alpha}{\beta}=1 \Rightarrow s_\beta(\alpha) = \alpha - \beta$$
	Which implies that $\alpha-\beta$ is a root (Axiom 3.)
	And if $\cartan{\beta}{\alpha}=1$ then $\beta-\alpha$ is a root and therefore $-(\beta-\alpha)=\alpha-\beta$ is a root (Axiom 1.).\\
	If $\langle \alpha,\beta \rangle$ is negative, then $\langle \alpha,-\beta \rangle $ is positive and therefore $\alpha-(-\beta) = \alpha+\beta$ is a root.
\end{proof}
Following this, we restrict ourselves to some specific bases of $E$ composed of roots based on Axiom $2.$.
\begin{defi}
	The set $R=\{v \in E| \langle \alpha,v \rangle \not=0 \text{ for all } \alpha \in \Phi\}$ is the set of regular elements.\\
	$$\Phi^+(\gamma) = \{\alpha \in \Phi | \langle\alpha, \gamma\rangle>0 \}\ \ \ \ \ \Phi^-(\gamma) = \{\alpha \in \Phi|\langle \alpha,\gamma \rangle < 0\}$$
	We call elements in $\Phi^+(\gamma)$ as positive roots and in $\Phi^-(\gamma)$ as negative roots with respective to $\gamma$.\\
	Furthermore, with respect to this ordering, we call a positive root decomposable if $\alpha=\beta_1+\beta_2$ for $\beta_1,\beta_2 \in \Phi^+(\gamma)$, and indecomposable if it's not decomposable.
\end{defi}
\begin{teo}[Root System Base]
	The set $\Delta(\gamma)$ of indecomposable elements is a base of $E$ and every element in $\Phi_+$ is in the $\mathbb{Z}_+$-span of $\Delta(\gamma)$.
	\label{basetheo}
\end{teo}
\begin{proof}
We proceed in steps:
\begin{enumerate}
	\item Every element in $\Phi_+$ is in the $\mathbb{Z}_+$-span of $\Delta(\gamma)$.\\
	Otherwise, let $\beta$ be such element that can't be written with $\langle \gamma,\beta \rangle$ as small as possible obviously $\beta \not \in \Delta(\gamma)$, but in that case $\beta=\beta_1+\beta_2$ for some $\beta_1,\beta_2 \in \Phi^+$, but since $$\langle \gamma,\beta \rangle = \langle \gamma,\beta_1 \rangle + \langle \gamma,\beta_2 \rangle$$
	And each of $\langle \gamma,\beta_1 \rangle$ and $\langle \gamma,\beta_2\rangle$ is positive, then $\beta_1,\beta_2$ must be in the $\mathbb{Z}_+$-span of $\Delta(\gamma)$ to avoid contradicting the minimality of $\beta$.\\
	But then $\beta = \beta_1+\beta_2$ is in the $\mathbb{Z}_+$ span of $\Delta(\gamma)$
	\item $\Delta(\gamma)$ spans $E$\\
	Since $\Phi$ spans $E$ and $\Phi=\Phi^+\cup\Phi^-$ then the previous point proves that $\Delta$ spans $\Phi$ and therefore spans $E$.
	\item If $\alpha,\beta$ are distinct elements in $\Delta(\gamma)$, then $\langle \alpha, \beta \rangle \le 0$.\\ Otherwise, since $\beta$ clearly can't be $-\alpha$, then $\alpha-\beta$ is in $\Phi$(Corollary \ref{corolcartan}`), therefore either $\alpha-\beta$ or $\beta-\alpha$ are in $\Phi^+$, but if $\alpha-\beta \in \Phi^+$ then $\alpha = \beta + (\alpha-\beta)$ and if $\beta-\alpha \in \Phi^+$ then $\beta = \alpha + (\beta-\alpha)$, contradicting the fact that $\alpha$ and $\beta$ are indecomposable.
	\item $\Delta(\gamma)$ is a linearly independent set
	Suppose not, then let $\{r_\alpha\}$ be such that $\sum_{\alpha \in \Delta} r_\alpha \alpha = 0$, divide this sum in the case in which $r_\alpha > 0$ (call it $\Delta_1$) and $r_\alpha<0$(call it $\Delta_2$), then:
	$$\sum_{\alpha \in \Delta_1}r_\alpha \alpha = \sum_{\beta \in \Delta_2}-r_\beta \beta = \epsilon \not=0$$
	But therefore $\langle \epsilon,\epsilon\rangle=\displaystyle\sum_{\alpha,\beta} -r_\alpha r_\beta\langle \alpha,\beta\rangle > 0$ bu $r_\alpha>0$,$r_\beta <0$ and $\langle \alpha, \beta \rangle \le 0$, a contradiction.\\
	\textbf{Remark:} $4.$ Implies that any set in an euclidean space that are simultaneously non-acute are linearly independent.
\end{enumerate}
\end{proof}\\
From this point forward we fix a basis $\Delta$ and call its elements simple.
\begin{lema}
	If $\alpha$ is a positive root but not simple, then $\alpha-\beta$ is a positive root for some $\beta \in \Delta$.\\
	In particular, every positive root can be written as $\alpha_1+\cdots +\alpha_i$ with each $\alpha_k \in \Delta$ not necessarily distinct in such a way that each partial sum is a root.
	\label{partialsum}
\end{lema}
\begin{proof}
	If $(\alpha,\beta)\le 0$ for every $\beta \in \Delta$ then the remark in Theorem \ref{basetheo} would apply and $\Delta \cup \{\alpha\}$ would be linearly independent, which is absurd.Therefore there exists a $\beta \in \Delta$ such that $\langle \alpha,\beta \rangle > 0$ and then $\alpha-\beta \in \Phi$, positive because the uniqueness in the decomposition of $\alpha$ imply that $\alpha = \sum_\Delta k_\gamma \gamma$ with some $k_\gamma > 0$ for $\gamma \not=\beta$, and subtracting $\beta$ would not change this coefficient.
\end{proof}
\begin{lema}
	Let $\alpha$ be a simple root, then $s_\alpha$ permutes all positive roots except $\alpha$
	\label{permutepositive}
\end{lema}
\begin{proof}
	Let $\beta \in \Phi^+ - \{\alpha\}$, then we can write 
	$$ \beta = \sum_{\gamma \in \Delta} k_\gamma \gamma \ \ \ k_\gamma \in \mathbb{Z}_+ \ \ k_i \not=0 \text{ for some }\\i\not=\alpha$$
	But then since $\alpha \in \Delta$ the coefficient $k_i$ is unchanged in $s_\alpha(\beta) = \beta - \cartan{\beta}{\alpha}\alpha$, and therefore $s_\alpha(\beta)$ has at least one positive coefficient and is therefore a positive root. Moreover since $s_\alpha$ is bijective and $s_\alpha(-\alpha)=\alpha$ then $s_\alpha(\beta)\not=\alpha$ 
\end{proof}
\begin{corol}
	Set $\delta = \displaystyle \frac{1}{2}\sum_{\beta \in \Phi^+} \beta$. Then $s_\alpha(\delta) = \delta - \alpha$ for all $\alpha \in \Delta$
	\label{corolexistancedelta}
\end{corol}
\begin{proof}
	Let $\Phi_\alpha = \Phi^+-\{\alpha\}$ and remember that $s_\alpha(\alpha)=-\alpha$
	\begin{align*}
		s_\alpha(\delta) &= \displaystyle \frac{1}{2}\sum_{\beta \in \Phi^+} s_\alpha(\beta)\\
		&= \displaystyle \frac{1}{2}\sum_{\beta \in \Phi_\alpha}  s_\alpha(\beta) + \frac{1}{2}s_\alpha(\alpha)\\
		&= \frac{1}{2}\sum_{\beta \in \Phi_\alpha} \beta - \frac{1}{2}\alpha\\
		&= \gamma - \frac{1}{2}\alpha - \frac{1}{2} \alpha = \gamma-\alpha
	\end{align*}
\end{proof}

One important subset of Root Systems are those that are irreducible in the sense that there are no sub-root systems, to keep things algebraically simple, we define them in the following way:
\begin{defi}[Irreducible Root Systems]
	\label{irredRS}
	A root system $\Phi$ is called reducible if there exists two proper subsets $\Phi_1$ and $\Phi_2$ of $\Phi$ with $\langle \Phi_1,\Phi_2 \rangle = 0$ and $\Phi_1 \cup \Phi_2 = \Phi$\\
	A root system will be called irreducible if it's not reducible.
\end{defi}
One important property of the irreducibility on root systems is that it can be reduced to a base:
\begin{prop}
	A root system $\Phi$ with base $\Delta$:\\
	If $\Delta$ can't be partitioned into two proper orthogonal subsets, then $\Phi$ is irreducible
	\label{irbaseimplyirroot}
\end{prop}
\begin{proof}
	Let $\Phi = \Phi_1 \cup \Phi_2$, then we can partition delta as: $\Delta = (\Delta \cap \Phi_1) \cup (\Delta \cap \Phi_2)$ which is a partition by orthogonal subsets, proper unless $\Delta$ is contained in one of them,  $\Phi_1$ without loss of generality, but that implies:
	$$(\Delta,\Phi_2) = 0 \Rightarrow (E,\Phi_2) = 0 \Rightarrow \Phi_2 = 0$$
	A contradiction, showing that if the base is irreducible then so is the root system.\\
	The converse is true, although the proof is not entirely trivial and shall not be covered for the purposes of this section.
\end{proof}
\section{Weyl Group}
The structure of root system is very deeply connected to that of reflection groups, and within any root system, there's a clear reflection group: The group generated by all root reflections
\begin{defi}
	The group generated by all of the root reflections is called the Weyl group:
	$$ \W = \langle s_\alpha | \alpha \in \Phi\rangle $$
	\label{Wdef}
\end{defi}
It is clear that this group is finite as a subgroup of root permutations, since reflections preserve the inner product, so does $\W$, and therefore $\W$ can be viewed as a finite subgroup of orthogonal transformations of $E$.\\
Another important point to consider is that $\W$ is normal in the group of automorphisms of $\Phi$, let $\Aut \Phi = \{T \in \gl(V)| T(\Phi)=\Phi \text{ and } \cartan{\alpha}{\beta} = \cartan{T(\alpha)}{T(\beta)}\}$
\begin{prop}
	If $T \in \Aut \Phi$ then $Ts_\alpha T^{-1} = s_{T(\alpha)}$
	\label{Wnormality}
\end{prop}
\begin{proof}
	Let $\beta \in \Phi$ be any root, then:
	$$Ts_\alpha T^{-1}(\beta) = Ts_\alpha(T^{-1}\beta) = T(T^{-1}\beta - \cartan{T^{-1}\beta,\alpha}\alpha) = \beta - \cartan{\beta}{T(\alpha)}T(\alpha) = s_{T(\alpha)}(\beta)$$
	Since $\Phi$ spans $E$, the result follows.
\end{proof}
\begin{lema}
	Let $\alpha_1,\cdots,\alpha_n \in\Delta$ (not necessarily distinct). Write $s_{\alpha_i}=s_i$. If $s_1\cdots s_{n-1}(s_n)$ is a negative root, then there exists some $1\le t<n$ such that $s_1 \cdots s_n = s_1 \cdots s_{t-1}s_{t+1} \cdots s_{n-1}$
\end{lema}
\begin{proof}
	Write $\beta_i = s_{i+1} \cdots s_{n-1}(\alpha_n)$ for $1\le i \le n-2$ and $\beta_{n-1} = \alpha_n$. Since $\beta_0$ is negative by hypothesis and $\beta_{n-1}$ is positive, we can find the smallest index $t$ such that $\beta_t$ is positive.\\
	But then $\sigma_t{\beta_t} = \beta_{t-1}$ which is negative, this forces $\beta_t = \alpha_t$ by Lemma \ref{permutepositive}, considering Proposition $\ref{Wnormality}$ for $T \in \W$ we have that:
	$$s_{T(\alpha)} = Ts_\alpha T^{-1}$$ 
	In particular since $\alpha_t = s_{t+1}\cdots s_{n-1}(\alpha_n)$
	$$s_t =  (s_{t+1}\cdots s_{n-1}) s_n (s_{n-1}\cdots s_{t+1})$$
\end{proof}
\begin{corol}
	If $s \in \W$ is written as a multiplication of simple roots $s=s_1\cdots s_t$, with $t$ as small as possible, then $s(\alpha_t)$ is negative.
	\label{Wcorolsimple}
\end{corol}
We want to prove that $\W$ is generated by a basis $\Delta(\gamma)$, for this consider the subgroup $\W'(\gamma)$ as $\langle s_\alpha | \alpha \in \Delta(\gamma)\rangle$, then the following is valid for $\W'(\gamma)$:
\begin{teo}
	If $\gamma' \in R$ is another regular element, then there exists $s \in \W'$ such that $\langle s(\gamma'),\alpha \rangle >0$ for all $\alpha \in \Delta$, and in particular there exists $s' \in \W'(\gamma)$ such that $s'(\Delta(\gamma)) = \Delta(\gamma')$ 
	\label{32conjugacychambers}
\end{teo}
\begin{proof}
	Consider $\delta$ as defined in Corollary \ref{corolexistancedelta} and choose $s \in \W'$ such that $\langle s(\gamma'),\delta\rangle $ is as big as possible.\\
	For any $\alpha \in \Delta(\gamma)$, then $ss_\alpha \in \W'$, and the choice of $s$ implies:
	\begin{align*}
		\langle s(\gamma'),\delta\rangle \ge \langle s_\alpha s(\gamma'),\delta\rangle  &= \langle s(\gamma'),s_\alpha(\delta)\rangle \\
		&= \langle s(\gamma'), \delta - \alpha \rangle \\
		&= \langle s(\gamma'),\delta \rangle - \langle s(\gamma'),\alpha\rangle 
	\end{align*}
	Which implies that $\langle s(\gamma'), \alpha \rangle \ge 0$, but it can't be $0$, since $\langle s(\gamma'),\alpha \rangle = \langle \gamma', s^{-1}(\alpha)\rangle$ and $\gamma'$ is regular.\\
	Now let $s'=s^{-1}$, we want to prove that $s'(\Delta(\gamma))$ is irreducible with respect to $\Phi^+(\gamma')$. Since $\langle \gamma', s'(\alpha) > 0$ for any $\alpha \in \Delta$, the same is true for $\Phi^+(\gamma)$ since they are in the $\mathbb{Z}_+$ of $\Delta(\gamma)$, therefore proving, since they have the same cardinality that $\Phi^+{\gamma'} = s'(\Phi^+(\gamma))$\\
	Now suppose that $s'(\Delta(\gamma))$ is not composed of irreducible roots with respect to $\gamma'$, then there exists an $\alpha \in \Delta(\gamma)$  and $\beta_1,\beta_2 \in \Phi_+{\gamma} $ such that $s'(\alpha) = s'(\beta_1 + \beta_2)$, but this in turn implies that $\alpha = \beta_1 + \beta_2$, contradicting the fact that $\alpha$ is an irreducible root.\\
\end{proof}
\begin{lema}
	Given any root $\alpha$, then there exists a regular element $\gamma$ such that $\alpha \in \Delta(\gamma)$.
	\label{32anyrootisbasic}
\end{lema}
\begin{corol}
	Given any root $\alpha$, then there exists $s \in \W'(\gamma)$ such that $s(\alpha) \in \Delta(\gamma)$
	\label{32conjugacyofroots}
\end{corol}
\begin{proof}
	Let $\gamma' \in R$ be such that $\alpha \in \Delta(\gamma')$ (Lemma \ref{32anyrootisbasic}) and let $s \in \W'$ be such that $s(\Delta(\gamma'))=\Delta(\gamma)$(Theorem \ref{32conjugacychambers}), then $s(\alpha) \in \Delta(\gamma)$
\end{proof} 
\begin{prop}
	$\W = \W'(\gamma)$ for any $\gamma \in R$´
\end{prop}
\begin{proof}
	We'll prove that $s_\alpha$ for $\alpha \in \Phi$ is an element of $\W'(\gamma)$, using Corollary \ref{32conjugacyofroots} then there exists $s \in \W'$ such that $s(\alpha) \in \Delta(\gamma)$, call $s(\alpha)=\beta$, then $s^{-1}(\beta)=\alpha$, but on the other hand(Proposition \ref{Wnormality}):
	$$s_\alpha = s_{s^{-1}\beta} = s^{-1}s_\beta s \in \W' $$
\end{proof}
\begin{prop}
	The only $s \in \W$ such that $s(\Delta)=\Delta$ in $\W$ is $1$
\end{prop}
\begin{proof}
	Suppose $s(\Delta)=\Delta$ but $s \not= 1$, then Corollary \ref{Wcorolsimple} implies that $s(\alpha_k)$ is negative for some $k$, a contradiction.
	
\end{proof}
\section{Weight Theory}
\chapter{Universal Lie Algebras and Conjugation}
\input{41Conjugation}
\section{Universal Enveloping Algebras}
\section{Poincaré-Birkhoff-Witt Theorem}
\chapter{Category $\bggO$}
\section{Axiomatic and Initial Properties}
\section{Verma Modules}
\section{Formal Characters}
\section{Harish-Chandra Theorem}
\chapter{Observações(EDITS TO DO)}
\begin{enumerate}
	\item Proof of preservation of Abs Jordan Decomposition between any representation and the geometric Lemma \ref{32anyrootisbasic}
	\item More Examples \ref{11EXAMPLES} \ref{MOREEXAMPLES} 
	\item Counter-example to Lie Theorem in prime characteristic
	\item Discussions and referencing.
	(\cite{jacobson}\cite{humphreys2}\cite{sanmartin})
\end{enumerate}
\bibliographystyle{alpha}
\bibliography{bibliography}
\end{document}
