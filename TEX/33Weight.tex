\section{Weight Theory}
One important of how root systems are applied to Lie theory is on the idea of how representations can be decomposed in weight spaces as we did with $\gsl(2)$, in this section we try to analyze these aspects individual of those facts, just by defining weights as their own structure inside root systems and the consequences of the fact, similar to the approach done in \cite{humphreys1} later connections between root systems and  are seen in the Representation Theory section.\\
\begin{defi}[Weight Sets]
	Given a root system $(\Phi,E)$ with a base $\Delta$, an element $\lambda \in E$ is said to be a weight if:
	$$ 2\frac{\langle \lambda , \alpha \rangle}{\langle \alpha,\alpha\rangle } = \cartan{\lambda}{\alpha} \in \mathbb{Z} \ \ \ \text{for all } \alpha \in \Delta$$
	The set of \textbf{all weights} is denoted by $\Lambda$ which is given a structure of group by being embedded in a vector space and being closed under addition. \\
	We call the $\textbf{root lattice}$ as the subgroup of $\Lambda$ generated by the root themselves denoted by $\Lambda_r = \langle \alpha, \alpha \in \Phi \rangle_\Lambda$.\\
	Finally if $\lambda$ is such that all Cartan products $\cartan{\lambda}{\alpha}$ for $\alpha \in \Delta$ are non-negative(respectively positive) integers we call it a \textbf{dominant weight}(respectively strongly dominant) and denote the set of dominant weights by $\Lambda^+$
\end{defi}
$\Delta = \{\alpha_1,\cdots,\alpha_l\}$ forms a basis of $E$ and so does the set $\Delta^\lor = \{\alpha_1^\lor, \alpha_2^\lor , \cdots, \alpha_l^\lor \}$, the dual basis(with respect to the inner product) of $\Delta^\lor$, denoted by $\{\lambda_1,\cdots,\lambda_l\}$ is defined by the relation:
$$ \cartan{\lambda_i}{\alpha_j} = \delta_{ij}$$
We call these elements \textbf{fundamental dominant weights} with respect to $\Delta$, called as such because of the following:
\begin{prop}
If $\lambda \in E$ is a weight, then $\lambda = \displaystyle\sum_{k=1}^l m_k\lambda_k$ where $m_k = \cartan{\lambda}{\alpha_k}\in \mathbb{Z}$. \\ 
In other words, $\Lambda$ is a lattice generated by $\{\lambda_1,\cdots,\lambda_l\}$
\end{prop}
\begin{proof}
For any $j \in \{1,\cdots,l\}$ we have that:
$$\cartan{\lambda}{\alpha_j} = m_j = \sum_{i=1}^l \delta_{ij}m_i = \sum_{i=1}^l m_i\langle  \lambda_i, \alpha_j^\lor \rangle = \left\langle \sum_{i=1}^l m_i\lambda_i, \alpha_j^\lor \right\rangle \Rightarrow \left\langle \lambda - \sum_{i=1}^l m_i\lambda_i,\alpha_j^\lor \right\rangle = 0$$
And therefore, since $\{\alpha_1^\lor, \cdots, \alpha_l^\lor\}$ forms a basis of $E$, $\lambda=\displaystyle\sum_{i=1}^l m_i\lambda_i$
\end{proof}\\
We want to define an ordering in $E$, one natural way to do so is as follows:For $\lambda,\mu \in E$ we say that $\lambda \prec \mu$ if $\mu - \lambda$ is a sum of simple roots or $\lambda = \mu$.\\
We can derive some properties of this definition directly:
\begin{lema}
	With the above notations:
	\begin{enumerate}
		\item $\alpha \in \Phi^+ \Rightarrow \alpha \succ 0$
		\item If $\lambda \in \Lambda^+$ then $s \lambda \prec \lambda$ for all $\sigma \in \mathcal{W}$
		\item If $\lambda$ is a strongly dominant weight then $s \lambda = \lambda$ for $s \in \mathcal{W}$ only if $s = 1$ 
		\item If $\lambda \in \Lambda^+$ then the number of dominant weights $\mu$ such that $\mu \prec \lambda$ is finite.
	\end{enumerate}
	\label{33orderingproperties}
\end{lema}
\begin{proof}
	\begin{enumerate}
		\item It's the definition of a positive root
		\item For the simple reflections $s_i$ note that:
		$s_i\lambda = \lambda - \cartan{\lambda}{\alpha_i} \alpha_i = \lambda - m_i \alpha_i$.\\
		Which implies that $\lambda - s_i\lambda = m_i \alpha_i \succ 0$, which is also a dominant weight.\\
		Just repeat the argument on the length of $\sigma$ since $\mathcal{W}$ is generated by the simple roots.
		\item Let $\lambda = \sum m_i \lambda_i$ for $m \in \mathbb{Z}_{>0}$, then if $\sigma \in \mathcal{W}$ is non-trivial then $s^{-1}(\alpha_j)$ is negative for some $j$(Corollary \ref{Wcorolsimple}), and therefore $\cartan{s\lambda}{\alpha_j}  =\langle \lambda, s^{-1}(\alpha_j^\lor) \rangle < 0$, to see this, let $s^{-1}(\alpha_j) = -\sum_{k=1}^l r_k \alpha_k$ with $r_k \ge 0$ but not all zero, then $s^{-1}(\alpha_j^\lor) = -\frac{2}{\langle \alpha_j,\alpha_j \rangle}\sum_{k=1}^l r_k\alpha_k$.
		$$ \langle \lambda, s^{-1}(\alpha_j^\lor)\rangle =  -\frac{2}{\langle \alpha_j,\alpha_j\rangle} \sum_{i,k} m_ir_k \langle \lambda_i,\alpha_k\rangle \le 0$$
		\item If $\mu \in \Lambda^+$ is such that $\mu \prec \lambda$ then since $\lambda + \mu \in \Lambda^+$ and $\lambda - \mu$ is sum of positive roots then:
		$$\langle\lambda + \mu, \lambda - \mu \rangle \ge 0 \Rightarrow  \langle \lambda, \lambda \rangle \ge \langle \mu, \mu \rangle $$
		Since a compact and discrete set is finite, $\Lambda^+$ is discrete and the set $\{\mu \ |\  \langle \mu, \mu \rangle  \le \langle \lambda, \lambda \rangle \}$ is compact then the set of possible $\mu$ is finite.
	\end{enumerate}
\end{proof}\\
On the other hand, some other properties aren't valid, for example it's possible that $\mu \prec \lambda$ for some $\mu$ dominant and $\lambda$ not dominant, and not all dominant weights $\mu$ satisfy $\mu \succ 0$.
\begin{ex}
	ADD EXAMPLE REFERENCING $A_2$.
	\label{refweight}
\end{ex}
\textbf{The weight $\delta$}\\
We remind ourselves of the weight $\delta$ defined in Corollary \ref{corolexistancedelta}
$$ \delta = \frac{1}{2}\sum_{\alpha \in \Phi^+} \alpha \ \ \ s_i\delta = \delta-\alpha_i$$
Some properties of this particular weight are relevant 
\begin{lema}
	$\delta = \displaystyle \sum_{i=1}^l \lambda_i$, meaning that $\delta$ is a strongly dominant weight, moreover, for all $\mu \in \Lambda^+$ and $s \in \mathcal{W}$  $$\langle s\mu + \delta, s\mu+ \delta\rangle \le (\mu + \delta, \mu + \delta)$$ with equality if and only if $s\mu = \delta$ 	
\end{lema}
\begin{proof}
	We know that $s_i\delta=\delta-\alpha_i$ which is equivalent to saying $\cartan{\delta}{\alpha_i}=1$, proving that $\delta$ is a strongly dominant weight, now since $\delta = \displaystyle\sum_{k=1}^l \cartan{\delta}{\alpha_k}\lambda_k$ then the equality follows.\\
	For the second part consider the following:
	\begin{align*}
	\langle s\mu + \delta, s\mu + \delta \rangle  = \langle \mu + s^{-1}\delta, \mu + s^{-1}\delta \rangle &= \langle \mu+ \delta + s^{-1}\delta - \delta , \mu + \delta + s^{-1}\delta - \delta \rangle \\
	&=\langle \mu + \delta, \mu + \delta\rangle - 2\langle \mu,\delta-s^{-1}\delta\rangle
	\end{align*}
	Since $\delta \in \Lambda^+$ then $\delta \prec s^{-1}\delta$ (Lemma \ref{33orderingproperties} 2.), then $\langle \mu, \delta - s^{-1}\delta\rangle \ge 0$ because $s-s^{-1}\delta$ is a sum of positive roots. The equality will hold if and only if:
	$$\langle\mu,\delta-s^{-1}\delta\rangle = 0 \iff \langle \mu,\delta\rangle = \langle \mu, s^{-1}\delta\rangle = \langle s \mu, \delta \rangle \iff \langle \mu - s\mu,\delta \rangle = 0$$
	But again $\mu-s\mu$ is a sum of positive roots, and $\delta$ is strongly dominant, concluding that this can only equal $0$ if $\mu=s\mu$
\end{proof}\\
\textbf{Saturated set of Weights}
Certain sets of weights play a predominant role in representation theory, what follows is a definition of those sets and some of their properties independently of Lie Algebras.
\begin{defi}
	A finite subset $\Pi$ of $\Lambda$ is called \textbf{saturated} if for all $\lambda \in \Pi,\ \alpha \in \Phi$ and $i$ between $0$ and $\langle \lambda, \alpha^\lor\rangle $, the weight $\lambda - i\alpha$ also lies in $\Pi$.\\
	We say that a saturated set of weights has \textbf{highest weight} $\lambda$ if it's a maximum of $\Pi$ with respect to $\prec$.\\
\end{defi}
\begin{lema}
	A saturated set of weights having highest weight $\lambda$ is finite 
\end{lema}
\begin{proof}
	Direct consequence of Lemma \ref{33orderingproperties} 4.
\end{proof}
\begin{prop}
	Let $\Pi$ be saturated, with highest weight $\lambda$. If $\mu \in \Lambda^+$ and $\mu\prec \lambda$ then $\mu \in \Pi$
\end{prop}
\begin{proof}
	Suppose $\mu' = \mu + \displaystyle	\sum_{\alpha \in \Delta} k_\alpha \alpha \in \Pi$ for some $k_\alpha \in \mathbb{Z}^+$, we shall prove that we can reduce $k_\alpha$ by one and still stay in $\Pi$, meaning that if we have a starting $\mu'$ we can get to $\mu$ by using only the definition of $\Pi$. Since $\mu \prec \lambda$ then $\lambda$ is such $\mu'$.\\
	If $\mu' \not= \mu$, (some $k_\alpha$ is non-zero) then: $\displaystyle\sum_{\alpha,\beta \in \Delta}\langle k_\alpha\alpha, k_\beta\beta\rangle > 0$ therefore $\displaystyle \sum_{\alpha \in \Delta} \langle k_\alpha\alpha,\beta \rangle >0$ for some $\beta \in \Delta$, since $\mu$ is dominant then $\cartan{\mu}{\beta} \ge 0$, and therefore $\cartan{\mu'}{\beta} > 0$. Concluding $\mu'-\beta \in \Pi$
\end{proof}
