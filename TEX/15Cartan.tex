\section{Cartan's Criteria}
Here we will consider a finite dimensional vector space $V$ over an algebraically closed field $\F$.
\begin{prop}
	If $X \in \gl(V)$ is such that its Jordan decomposition(Proposition \ref{jordandecom}) is $X=S+N$ then $\ad X = \ad S + \ad N$ is the Jordan decomposition of $\ad X$ in $\gl(\gl(V))$.
\end{prop}
\begin{proof}
	A basis of $\gl(V)$ is $e_{ij}$ with respect to a basis such that $S= \text{diag}{\lambda_i}$ then $[S,e_{ij}]=(\lambda_i-\lambda_j)e_{ij}$, therefore $\ad(S)$ is semi-simple, if $N$ is nilpotent then $\ad(N)$ is nilpotent (Proposition \ref{propadnilpotency}). Since the adjoint is a representation, then $[\ad S,\ad N]_{\gl(\gl(V))} = \ad[S,N] = 0$ and then the result follows from the uniqueness of Jordan Decomposition in $\gl(\gl(V))$.
\end{proof}
\begin{teo}[Cartan's Lemma]
	Let $A \subset B$ be two subspaces of $\gl(V)$ with $\dim V < \infty$ over a field with characteristic $0$, then set $M=\{X \in \gl(V)\ |\ [X,B]\subset A\}$. If $X \in M$ is such that $\Tr(XY)=0$ for all $Y \in M$. Then $X$ is nilpotent.
	\label{Cartan's Lemma}
\end{teo}
\begin{proof}
	This proof consists mostly of technicalities on proving that specific matrices are present in $M$.\\ 
	Let $X=S+N$ be the Jordan decomposition of $X$, fix a basis $\{v_1,\cdots,v_m\}$ in which $S=\text{diag}(a_1,\cdots,a_m)$. We want to prove that $a_1=\cdots=a_m=0$ since in that case $S=0$ and $X$ is nilpotent under the hypothesis.\\
	Let $E$ be the subspace of $\F$ generated by the $a_i$ over $\mathbb{Q}$ since we are assuming $\text{char } \F= 0$, then it is enough to show that $E=0$, or equivalently $E^*=0$.\\
	Let $f$ be a linear functional, and $Y=\text{diag}(f(a_1),\cdots,f(a_m))$. If $\{e_{ij}\}$ is the corresponding basis of $\gl(V)$, then $\ad(S)(e_{ij})` = (a_i-a_j)e_{ij}$ and $\ad(Y)(e_{ij}) = (f(a_i)-f(a_j))e_{ij}$. Now let $r(x)\in\F[X]$ be the polynomial such that $r(a_i-a_j)=f(a_i)-f(a_j)$, the existence of which follows from the Lagrange polynomial and linearity. Then $\ad Y = r(\ad S)$.\\
	Now $\ad S$ is the semi-simple part of $\ad X$, so it can be written as a polynomial in $\ad X$ without constant term by (Proposition `\ref{jordandecom}B.) and therefore $\ad Y$ as a polynomial in $S$ maps $B$ to $A$, proving that $Y$ is in $M$.\\
	Now $\Tr(XY)=\displaystyle\sum_{j=0}^m a_jf(a_j)=0 \Rightarrow \displaystyle\sum_{j=0}^m f(a_j)^2 =0$ by applying $f$ to the equality which in turn, since we restricted ourselves to $\mathbb{Q}$, implies that $f(a_j)=0$ for all $j$ and therefore $f=0$. Since $f$ is arbitrary then $E^*=0$.
\end{proof}	\\
This Lemma, very technical and apparently not very useful is essential to one of the most important facts about Lie Algebras as means of characterization of semi-simple Lie Algebras, in fact:
\begin{teo}[Cartan's Criterion]
	Let $\g$ be a subalgebra of $\gl(V)$, $V$ finite dimensional over a field $\F$ with characteristic $0$. Suppose that $\Tr(XY)=0$ for all $X \in [\g,\g]$ and $Y \in \g$. Then $\g$ is solvable.
	\label{Cartan's Criterion}
\end{teo}
\begin{proof}
	Let $A=[\g,\g]$ and $B=\g$, then the hypothesis shows that $\Tr(XY)=0$ for all $X \in A$ and $Y \in B$. We need a stronger statement to use the lemma, which is that for all $X \in A$ and $Y \in M$ it follows that $\Tr(XY)=0$, where $M=\{X \in \gl(V)| [X,B]\subset A\}$.\\
	Consider that $\Tr([X,Y]Z)=\Tr(X[Y,Z])$ for all $X,Y,Z \in \gl(V)$, since $$\Tr(XYZ - YXZ) = \Tr(XYZ)-\Tr(YXZ) = \Tr(XYZ)-\Tr(Y(XZ)) = \Tr(XYZ-XZY) = \Tr(X[Y,Z])$$
	Then let $[X,Y] \in A$ with $X$ and $Y \in \g$ and $Z \in M$, then:$\Tr([X,Y]Z)=\Tr(X[Y,Z])=0$ since $[Y,Z] \in A$ and $X \in B$. Therefore by Theorem \ref{Cartan's Lemma} every element in $[\g,\g]$ is nilpotent, which implies that $[\g,\g]$ is a nilpotent algebra, and therefore since $\g^{(n)}\subset \g^n = [\g,\g]^{n-1}=0$ for some $n$, $\g$ is solvable.
\end{proof}\\
This in turn allows us to classify algebras with respect to the trace form in the adjoint, in fact, let:
\begin{corol}
	Let $\g$ be any finite dimensional Lie Algebra over a field $\F$ algebraically closed with characteristic $0$, then if $\Tr(\ad(X) \ad(Y))=0$ for all $X \in [\g,\g]$ and $Y \in \g$. Then $\g$ is solvable.
	\label{Corollary}
\end{corol}
\begin{proof}
	We prove that $\ad \g = \g/\ker \ad$ is solvable from the theorem above, since $\ker \ad = \z(g)$ is solvable then $\g$ is solvable.\\
	It only remains to prove the following lemma: If $\h$ is a solvable ideal of $\g$ and $\g/\h$ is solvable, then $\g$ is solvable. To prove that, consider that $\h^{(n)} = 0$ for some $n$ and $(\g/\h)^{(m)} = [0]$ for some $m$, then $(\g^{(m)})^{(n)} \subset \h^{(n)} = 0$
\end{proof}\\
With this in mind, define the natural bilinear form in Finite Dimensional Lie Algebras, the Trace Form or Cartan-Killing form.
\begin{defi}
	Let $\g$ be any Lie Algebra. Define $\kappa(X,Y)= \Tr [\ad(X)\ad(Y)]$ in $\gl(\g)$ for $X$ and $Y$ in $\g$, then $\kappa$ is a symmetric bilinear form of $\g$, called the Killing form and is also invariant in the sense that $\kappa([X,Y],Z)=\kappa(X,[Y,Z])$.\\
	Its radical, called $\Rad \g = \{X \in \g | \kappa(X,Y)=0 \text{ for all } Y \in \g\}$ is an ideal of $\g$. 
	\label{Killing Form}
\end{defi}
With this in mind we can make the first step to classify semi-simple finite dimensional Lie Algebras.
\begin{teo}
	Let $\g$ be a finite dimensional Lie Algebra over $\F$ algebraic closed and over a field of characteristic $0$. Then the following are equivalent:
	\begin{enumerate}
		\item $\g$ is semi-simple.
		\item $\g$ has no non-zero abelian ideals.
		\item The Killing Form $\kappa(X,Y)=\Tr(\ad(X)\ad(Y))$ is non-degenerate
		\item $\g$ is a unique sum of simple ideals $\g_i$.
	\end{enumerate}
\end{teo}
\begin{proof}\\
	$1 \Rightarrow 2$, to prove this we will use a simple lemma, let $\h \ideal \g$, then $\h^{(k)}$ is an ideal of $\g$ for all $k$:\\
	By induction, basis case $k=1$ is the hypothesis, let $[X,Y] \in \h^{n}$, with $X,Y \in \h^{(n-1)}$ and $Z \in \g$, then:
	$$[Z,[X,Y]] = -[X,[Y,Z]]-[Y,[Z,X]] = [[Y,Z],X] + [[Z,X],Y]$$
	and by induction $[Y,Z],[Z,X]\in \h^{(n-1)}$ therefore $[Z,[X,Y]] \in [\h^{(n-1)},\h^{n-1}] = \h^{(n)}$.\\
	Finally if $\h$ is solvable with $\h^{(n)}=0$, then $\h^{(n-1)}$ is abelian, and by the previous discussion, it is an abelian ideal of $\g$.\\
	$2 \Rightarrow 3$, we will prove a more general result, that any abelian ideal contains the radical of the Killing form. In fact let $\h$ be an abelian ideal, then for all $X \in \h$ and $Y \in \g$ we get $\ad X \ad Y : \g \rightarrow \h$, which in turn implies that $(\ad X \ad Y)^2 : \g \rightarrow [\h,\h] = 0$, and then the operator $\ad X \ad Y$ is nilpotent, which implies that it has trace $0$, proving that $\kappa(X,Y)=0$ for all $X \in \h$ and $Y \in \g$, then by Cartan's lemma $\h$ is solvable and therefore if the Killing form has a non-zero radical then it has an abelian ideal.\\
	$3 \Rightarrow 4$, since $\kappa$ is a non-degenerate bilinear form, given an ideal $\h$ let $$\h^\perp = \{X \in \g | \kappa(X,Y)=0 \text{ for all } Y \in \h\}$$ we get $\g = \h \oplus \h^\perp$, $\h^\perp$ being an ideal of $\g$ by invariance, in fact given $X\in \h^\perp$ and $Z \in \g$ then for all $Y \in \h$ we get: $\kappa([X,Z],Y) = \kappa(X,[Z,Y])=0$ since $\h$ is an ideal.\\
	With this, we proceed the proof of existence. If $\g$ is simple, then $\g$ is the only ideal, and therefore it's done, otherwise, $\g$ has an ideal $\h$, and therefore $\g = \h \oplus \h^\perp$, $\h^\perp$ being a non-zero ideal of $\g$, the result follows by doing the same argument for $\h$ and $\h^\perp$ and from the fact that $\g$ is finite-dimensional.\\
	To prove uniqueness, assume the existence of a decomposition $\g = \bigoplus \g_i$, and let $\h$ be any simple ideal of $\g$, then from the fact that $\h$ is an ideal $[\g,\h]$ is an ideal contained in $\h$ and it is nonzero because otherwise $\h\subset \z(\g)=0$, and since $\h$ is simple then $[\g,\h]=\h$. On the other hand $[\g,\h]=\bigoplus[\g_i,\h]$, but since $[\g,\h]=\h$ is simple, then all terms are $0$ except one, proving that $\h=\g_j$ for some $j$.\\
	$4 \Rightarrow 1$ Let $\g= \bigoplus_{i \in I} \g_i$ be the decomposition of $\g$ in simple ideals, then let $\h$ be any ideal of $\g$, therefore $[\g,\h] = \bigoplus_{i \in I} [\g_i,\h]$, but for any specific $i$, then $[\g_i,\h]\ideal \g_i$, therefore it's either $0$ or $\g_i$, implying that $\h$ is a sum of $\g_i$, when $i$ is running on a subset of $I$, the result follows from induction on the size of the decomposition, since when $\g$ is simple, then the ideal can't be solvable.
\end{proof}