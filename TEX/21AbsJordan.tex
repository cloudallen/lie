\section{Abstract Jordan Decomposition}
There is a natural way to extend the Jordan decomposition of finite-dimensional operators to a given semi-simple Lie algebra by the adjoint representation, the main idea is proving the existence of elements in a Lie algebra satisfying the conditions of the decomposition.\\
\begin{lema}
Every derivation in a semi-simple Lie algebra is inner, meaning that if $D$ is a derivation then there exists $Y \in \g$ such that $D=\ad(Y)$
\label{InnerDerivations}
\end{lema}
\begin{proof}
Given $D$ a derivation, then the linear functional in $\g^*$ given by $f(X) = \Tr(D\ad(X))$ has a representational element $Y \in \g$ in such a way that $\kappa(X,Y) = f(X)$. And therefore $D=\ad(Y)$ because the element $\tilde{D} = D - \ad(Y)$, this relation implies $\Tr(\tilde{D}\ad(X))=0$ for any $X \in \g$. Now taking $X,Z \in \g$ arbitrary, then since:
$$[D,\ad(X)](Z) = D(\ad(X)(Z)) - \ad(X)DZ = D[X,Z] - [X,DZ] = [DX,Z] = \ad(DX)(Z)$$
Then:
\begin{align*}
\kappa(EX,Z) &= \Tr(\ad(EX)\ad(Z)) = \Tr([E,\ad(X)]\ad(Z))\\
&= \Tr(E\ad(X)\ad(Z) - \ad(X)E\ad(Z)) = \Tr(E\ad[X,Z])\\
&= 0
\end{align*}
Therefore since $Z$ is arbitrary $EX=0$ for any $X$ which implies that $E=0$, finally $D=\ad(Y)$.
\end{proof}
\begin{lema} 
	If $D$ is a derivation and $\lambda,\mu\in \F$ then for every $X,Y \in \g$:
	$$(D-(\lambda+\mu)I)^n[X,Y] = \sum_{i=0}^n {n \choose i}[(D-\lambda I)^{n-i}X,(D-\mu I)^iY].$$
	\label{LeibnizProductFormula}
\end{lema}
\begin{proof}
	Proceeding by induction, the basis being the case $n=1$: 
	\begin{align*}
	(D-(\lambda+\mu)I)[A,B] &= D[A,B] - (\lambda + \mu)[A,B] \\ 
	&= [DA,B] + [A,DB] - [\lambda A,B] - [A,\mu B]\\
	&= [(D-\lambda I)A,B] + [A,(D-\mu I)B].
	\end{align*}
	\begin{align*}
	(D-(\lambda + \mu)I)
	(D-(\lambda+\mu)I)^n[X,Y] &= (D-(\lambda+\mu)I)\sum_{i=0}^n {n \choose i}[(D-\lambda I)^{n-i}X,(D-\mu I)^iY]\\
	&=\sum_{i=0}^n {n\choose i}[(D-\lambda I)^{n-i+1}X,(D-\mu I)^i Y]\\
	&+\sum_{i=0}^n {n \choose i}[(D-\lambda I)^{n-i}X,(D-\mu I)^{i+1}Y]\\
	&=\sum_{i=1}^{n} \left({n \choose {i}}[(D-\lambda I)^{n+1-i},(D-\mu I )^i Y]\right) + [(D-\lambda I)^{n+1}X,Y]\\
	&+\sum_{i=1}^{n}\left({n \choose i-1}[(D- \lambda I)^{n+1-i},(D-\mu I)^i Y] \right) + [X,(D- \mu I)^{n+1}Y]\\
	&= \sum_{i=0}^{n+1}{{n+1}\choose i}[(D-\lambda I)^{n+1-i}X,(D-\mu I)^i Y].
	\end{align*}
\end{proof}
\begin{corol}
	On an algebraically closed field, if $D=S+N$ is the Jordan decomposition of a derivation, then $S$ and $N$ are derivations
\end{corol}
\begin{proof}
	Let $\g = \displaystyle\bigoplus_{\alpha \in \F} \g_\alpha$ be the generalized eigenspace space decomposition of $\g$, then the formula above shows that $[\g_\alpha,\g_\beta] \subset \g_{\alpha+\beta}$, and therefore if $X \in \g_\alpha$ and $Y \in \g_\beta$ then:
	$$S[X,Y] = (\alpha+\beta)[X,Y] = \alpha[X,Y]+\beta[X,Y] = [SX,Y]+[X,SY].$$
	And therefore $S$ is a derivation since $\g$ is the sum of eigenspaces, it follows that $N=D-S$ is a derivation.
\end{proof}
\begin{prop}
	For every $X \in \g$ there exists unique $S,N \in \g$ satisfying the following conditions:
	\begin{enumerate}[label=\alph*)]
		\item $X=S+N$
		\item $\ad\  S$ is diagonizable and $\ad\  N$ is nilpotent
		\item $[S,N]=0$
	\end{enumerate}
	\label{AbsJordanDecomp}
\end{prop}
\begin{proof}
Since $\ad(X)$ is a derivation, then its semi-simple part and nilpotent part are derivations and therefore are adjoints of elements in $\g$, let those elements be $S$ and $N$, $\ad(X)=\ad(S)+\ad(N)$.\\
Since $\g$ is semi-simple, then the adjoint representation is one-to-one(its kernel is $\z(\g)=0$) and therefore $\ad(X)=\ad(S+N) \Rightarrow X=S+N$\\
Finally, since $[\ad(S),\ad(N)]=0$ then $\ad[S,N]=0$ and therefore $[S,N]=0$.
\end{proof}
\begin{prop}
	If $\g$ is a semi-simple Lie algebra and $\varphi:\g \rightarrow \gl(V)$ is a representation, then for any element $X \in \g$ with Abstract Jordan Decomposition $X=S+N$ then the Jordan Decomposition of $\varphi(X)$ is $\varphi(X)=\varphi(S)+\varphi(N)$.\\
	In particular the representation of any semi-simple element is semi-simple and of every nilpotent element is nilpotent.
	\label{PreservationOfAbsJordan} 
\end{prop}
\begin{proof}
	Let's prove that given a semi-simple $\g \subset \gl(V)$ then the abstract and usual Jordan decomposition coincide, which reduces to proving that $\g$ contains the usual semi-simple and nilpotent parts of all its elements because both decompositions are unique.\\
	If $X \in \g$ is an arbitrary element, with usual Jordan decomposition given by $X=X_S+X_N$ in $\gl(V)$.If for a subspace $W$ of $V$ we have: $X(W)\subset W$ then $X_S(W)$ and $X_N(W)$ are also contained in $W$ (Proposition\ref{jordandecom}C).\\
	If $W$ is a sub-module of $V$, let $\g_W = \{Y \in \gl(V)|Y(W)\subset W \text{ and } \Tr(Y|_{W})=0\}$, for example $\g_V = \gsl(V)$. Let $\tilde{\g}=\displaystyle\bigcap_{W\le V} (N_{\gl(V)}(\g) \cap \g_W)$ where $N_{\gl(V)}(\g) = \{Y \in \gl(V)| [Y,\g]\subset \g\}$ we know that $\g\subset \tilde{\g}$ since $W$ is a sub-module and every element in $\g$ is a commutator $\g=[\g,\g]$ meaning that its trace is $0$`.\\
	For any $X \in \g$ then $X_S,X_N \in \tilde{\g}$ since their trace has to be $0$ in any $W\le V$ and they preserve $W$, now it remains to prove that $\g=\tilde{\g}$ to prove our result.\\
	Since $\tilde{\g}$ is a finite dimensional $\g$-module then the inclusion $\g \subset \tilde{\g}$ allows us to write $\tilde{\g} = \g \oplus M$. $[\g,\tilde{\g}]\subset \g$(Theorem $\ref{Weyl's Theorem}$). Finally let $W$ be an irreducible sub-module of $V$, then the fact that $[\g,\tilde{\g}]\subset \g$ implies that $[\g,Y]=0$ for $Y \in M$, this implies that $Y$ acts on $W$ by scalars (Lemma \ref{Schur's Lemma}), on the other hand since $\Tr(Y|_W)=0$ then $Y=0$ and therefore $\g = \tilde{\g}$.
\end{proof}
`