\section{Axiomatic}
We begin the study of category $\mathcal{O}$ by reminding ourselves of the equivalence between $\g$ and $U(\g)$ modules, every $\g$ module induces an $U(\g)$ module by the universal property and every $U(\g)$ module gives rise to a $\g$-module through the action of the length one elements.\\
On this chapter fix a finite dimensional Lie Algebra $\g$, a Cartan subalgebra $\h$ and a basis $\Delta$
The importance of weight modules to the study of representations are clear through their properties remarked by the standard cyclic modules given in the previous section.\\
\begin{defi}
We will say that a $U(\g)$-module is on BGG category $\bggO$ if it satisfies the following conditions:
\begin{enumerate}[label=($\bggO$\arabic*)]
	\item $M$ is finitely generated as a $U(\g)$-module
	\item $M$ is a weight module
	\item For each $v \in M$ the subspace generated by $\n_+$ by $v$ is finite.
\end{enumerate}
\end{defi}
Since every finite-dimensional module is a weight module then they are trivially modules in $\bggO$, it is also easily seen that the irreducible standard cyclic modules are also in $\bggO$, from now on we shall call those highest weight modules based on the uniqueness of the highest weight in the simple case.\\
To examine its categorical properties, we consider objects the modules satisfying conditions $(\bggO1)$, $(\bggO2)$ and $(\bggO3)$. The morphisms are the ones inherited from $U(\g)$-mod, since we define it this way we say $\bggO$ is a \textbf{full subcategory} of $U(\g)$-mod as all arrows from two elements in $\bggO$ in the category $U(\g)$ are also in $\bggO$.\\
 