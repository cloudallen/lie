\section{Universal Enveloping Algebras}
The universal enveloping algebra of a Lie algebra is one of the basic tools in representation theory, it is an associative algebra that contains $\g$ and extends all its representations.
\begin{defi}[Universal Enveloping Algebra]
	A pair $(U,i)$, $U$ being an associative algebra with unity and $i:\g \rightarrow U$ is a representation is
	 an universal enveloping pair of a Lie Algebra $\g$ if:
	\begin{enumerate}
		\item It's enveloping, meaning that $i:\g \rightarrow U$ is injective and its image $i(\g)$ generates $U$ as an algebra
		\item It's universal, meaning that if $\rho:\g \rightarrow \gl(V)$ is a representation then there exists a unique $\tilde{\rho}:U \rightarrow \gl(V)$ morphism of associative algebras satisfying:
		$$ \tilde{\rho}(i(X)) = \rho(X) \text{ for all }X \in \g$$
	\end{enumerate}
	\begin{center}
	\begin{tikzcd}
		U \ar[dr,tail,"\tilde{\rho}"] \\
		\g \ar[u,"i"]\ar[r,swap,"\rho"]& \gl(V)
	\end{tikzcd}		
	\end{center}
\end{defi}
The idea of an universal object implies uniqueness, so before constructing the enveloping algebra explicitly, let's check uniqueness
\begin{remark}
	If there are two pairs $(U,i)$ and $(V,j)$ that are universal and enveloping then there is bijective morphism between then:
\end{remark}
\begin{center}
	\begin{tikzcd}
		& U \arrow[d,tail,"\tilde{j}"] \\
		\g \arrow[ur,"i"]\arrow[r,"j"]\arrow[dr,"i"] & V \arrow[d,tail,"\tilde{i}"] \\ 
		& U
	\end{tikzcd}
\end{center}
The idea to construct the universal enveloping algebra for any Lie Algebra comes from the fact that the Tensor Algebra is an universal associative algebra for a vector space, so we just need to reduce its structure to contain the Lie Algebra bracket in some way
\begin{prop}
	If $\g$ is a Lie Algebra and $I$ is the ideal generated by $$\{[X,Y]-X\otimes Y + Y \otimes X \ | \ X,Y \in \g\}$$ in the tensor algebra $\mathcal{T}(\g)$ then $U(\g)=\mathcal{T}(\g)/\mathcal{I}$ is an universal enveloping algebra of $\g$.
\end{prop}
\begin{proof}
	Notice that $U(\g)$ contains at least the scalars since the ideal only contains elements of order higher than $1$.\\
	If $\pi : \mathcal{T}(\g) \rightarrow U(\g)$ is the canonical projection then we define $i$ to be the restriction of $\pi$ to $\g \subset \mathcal{T}(\g)$.
	If $\rho: \g \rightarrow \gl(V)$ is any representation, then let $\varphi$ be the morphism from $\mathcal{T}(\g) \rightarrow \gl(V)$ extending $\rho$. Now since $\rho$ is a representation then for an element in $\mathcal{I}$ we get
	$$ \varphi(X \otimes Y - Y \otimes X - [X,Y])= \rho(X)\rho(Y) - \rho(Y)\rho(X) - \rho([X,Y])=0,$$
	therefore $I \subset \ker \varphi$ and $\varphi$ induces a morphism $\tilde{\rho}:U(\g)\rightarrow \gl(V)$ satisfying $ \tilde{\rho} \circ i = \rho$
	\begin{center}
		\begin{tikzcd}
			& U(\g) \arrow[tail,"\tilde{\rho}"]{dr}& \\
			\g\arrow[bend right=60,"\rho"]{rr} \arrow[ur,"i"] \arrow[hook]{r}  & T(\g) \arrow[tail,"\pi"]{u} \arrow [tail,"\varphi"]{r} & \gl(V) 
		\end{tikzcd}
	\end{center}
\end{proof}